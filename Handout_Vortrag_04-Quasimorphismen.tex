\documentclass[11pt,a4paper,ngerman,DIV=11]{scrartcl}

%%%%%%%%%%%%%%%%%%%%%%%%%%%%%%%%%%%%%%%%%%%%%%%%%%%%%%%%%%%%%%%%%%%%%
%%% packages
%%%%%%%%%%%%%%%%%%%%%%%%%%%%%%%%%%%%%%%%%%%%%%%%%%%%%%%%%%%%%%%%%%%%%

\usepackage[utf8]{inputenc}
\usepackage[T1]{fontenc}
\usepackage[ngerman]{babel}

\usepackage{amsmath}
\usepackage{amssymb}
\usepackage{amsthm}
\usepackage{mathtools}
\usepackage[all]{xy}
\usepackage{tikz}

\usepackage[babel]{csquotes}
\usepackage[shortlabels]{enumitem}
\usepackage[numbers,sort&compress]{natbib}
\usepackage{ifmtarg}
\usepackage{xstring}
\usepackage{remreset}

\usepackage[lowtilde]{url}
\usepackage[pdftex,bookmarks,colorlinks=false,pdfborder={0 0 0},%
            pdftitle={Seminar Beschränkte Kohomologie - %
                      Handout zu Vortrag 4: Quasimorphismen},%
            pdfauthor={Johannes Prem}]{hyperref}
%
\usepackage{cleveref}
\let\cref=\Cref

\usepackage{myhelpers}  % my own myhelpers.sty
\usepackage{mymathmisc} % my own mymathmisc.sty


%%%%%%%%%%%%%%%%%%%%%%%%%%%%%%%%%%%%%%%%%%%%%%%%%%%%%%%%%%%%%%%%%%%%%
%%% macro definitions and other things
%%%%%%%%%%%%%%%%%%%%%%%%%%%%%%%%%%%%%%%%%%%%%%%%%%%%%%%%%%%%%%%%%%%%%

% set the url style (especially for the bibliography)
\urlstyle{sf}

% make parenthesized versions of \ref and cleveref's \cref
\newcommand*{\pref}[1]{(\ref{#1})}
\newcommand*{\pcref}[1]{(\cref{#1})}

% make a even more clever \mycref that produces "Lemma 42a)" etc.
\newcommand{\mycref}[1]{%
    \begingroup%
    \StrCount{#1}{:}[\mycrefCount]%
    \StrBefore[\mycrefCount]{#1}{:}[\myrefMain]%
    \expandafter\cref\expandafter{\myrefMain}\,\ref{#1}%
    \endgroup%
}

% make \varepsilon and \varphi default
\varifygreekletters{\epsilon\phi}

% change the qedsymbol to my favoured blacksquare
\renewcommand{\qedsymbol}{$\blacksquare$}

% style for /all/ theorem like environments
\newtheoremstyle{mythms}
 {5pt}% space above
 {5pt}% space below 
 {}% body font
 {}% indent amount
 {\bfseries}% theorem head font
 {.}% punctuation after theorem head
 {0.3cm plus 0.05cm minus 0.1cm}% space after theorem head (\newline possible)
 {}% theorem head spec 
 
% set style and define thm like environments
\theoremstyle{mythms}
\newtheorem{globalnum}{DUMMY DUMMY DUMMY}[section]
\newtheorem{thDef}[globalnum]{Definition}
\newtheorem{thNotation}[globalnum]{Notation}
\newtheorem{thKonstruktion}[globalnum]{Konstruktion}
\newtheorem{thSatz}[globalnum]{Satz}
\newtheorem{thProposition}[globalnum]{Proposition}
\newtheorem{thLemma}[globalnum]{Lemma}
\newtheorem{thKorollar}[globalnum]{Korollar}

\newtheorem{thAufgabe}[globalnum]{Aufgabe}
\newtheorem{thSetup}[globalnum]{Setup}
\newtheorem{thErinnerung}[globalnum]{Erinnerung}
\newtheorem{thErinnerDef}[globalnum]{Erinnerung/Definition}
\newtheorem{thBemerkung}[globalnum]{Bemerkung}
%\newtheorem{thWarnung}[globalnum]{Warnung}
\newtheorem{thBeispiel}[globalnum]{Beispiel}
\newtheorem{thBeispiele}[globalnum]{Beispiele}
\newenvironment{BspList}[1][]{%
\nopagebreak\begin{thBeispiele}#1%
\hfill\begin{enumerate}[(a),parsep=0pt,itemsep=0.8ex,leftmargin=2em]%
}{%
\end{enumerate}\end{thBeispiele}
}
%

% also define a 'proofsketch' version of 'proof'
\newenvironment{proofsketch}[1][]{%
\begin{proof}[Beweisskizze#1]
}{%
\end{proof}
}

% inject pdfbookmarks at thm like environments
\makeatletter
\let\origthmhead=\thmhead
\renewcommand{\thmhead}[3]{%
\origthmhead{#1}{#2}{#3}%
\belowpdfbookmark{#1\@ifnotempty{#1}{ }#2\thmnote{ (#3)}}{#1#2}%
}
\makeatother

% new math operators
\DeclareMathOperator*{\bigdotcup}{\overset{\mkern0mu\scalebox{0.6}{$\bullet$}}{\bigcup}}
\DeclareMathOperator*{\freeprod}{\bigstar}

% avoid "already defined"
\let\ker=\relax
% new math 'operators'
\newcommand{\sDMO}[1]{\expandafter\DeclareMathOperator\csname#1\endcsname{#1}}

\sDMO{const}
\sDMO{Hom}
\sDMO{id}
\sDMO{im}
\sDMO{ker}
\sDMO{sgn}
\sDMO{QM}

% categories
\newcommand{\MakeCategoryName}[1]{%
    \expandafter\DeclareMathOperator\csname#1\endcsname{\mathsf{#1}}
}

\MakeCategoryName{Ab}
\MakeCategoryName{Ch}
\MakeCategoryName{CoCh}
\MakeCategoryName{Grad}
\MakeCategoryName{Group}
\MakeCategoryName{Mod}
\MakeCategoryName{Ring}
\MakeCategoryName{Set}
\MakeCategoryName{Top}
\MakeCategoryName{Vect}

%
\newcommand{\lXX}[2]{\mathop{{}_{#2}\mkern-2.5mu#1}}
\newcommand{\makeLRcat}[1]{%
    \expandafter\newcommand\csname l#1\endcsname{\expandafter\lXX\csname#1\endcsname}
    \expandafter\newcommand\csname r#1\endcsname[1]{\csname#1\endcsname_{##1}}
}
%
\makeLRcat{Mod}
\makeLRcat{Ch}
\makeLRcat{CoCh}
\makeLRcat{Grad}

% make quantors that use \limits per default
\DeclareMathOperator*{\Exists}{\exists}
\DeclareMathOperator*{\forAll}{\forall}

% define an 'abs', 'norm' and 'Spann' command
\DeclarePairedDelimiter{\abs}{\lvert}{\rvert}
\DeclarePairedDelimiter{\norm}{\lVert}{\rVert}
\DeclarePairedDelimiter{\Spann}{\langle}{\rangle}

% define missing arrows
\newcommand{\longto}{\longrightarrow}
\newcommand{\longhookrightarrow}{\lhook\joinrel\relbar\joinrel\rightarrow}
\newcommand{\isorightarrow}[1][]{\xrightarrow[#1]{\smash{\raisebox{-2pt}{$\sim$}}}}

% provide mathbb symbols \N \Z \Q \R and \C
\defmathbbsymbols{Z Q C}
\defmathbbsymbolsubs{N R}

% define some set specific macros
\newcommand{\setclosure}[1]{\overline{#1}}
\newcommand{\setinterior}[1]{#1^\circ}
\newcommand{\setboundary}[1]{\partial #1}

% quotient by means of groups/rings/vector spaces
\newcommand{\Quot}[3][\big]{%
\raisebox{2pt}{$\mathsurround=0pt\displaystyle #2$}\mkern-3mu%
#1/%
\mkern-3mu\raisebox{-3.5pt}{$\displaystyle #3$}%
}
\newcommand{\QuotS}[3][]{%
\raisebox{2pt}{$\mathsurround=0pt\displaystyle #2$}\mkern-1mu%
#1/%
\mkern-3mu\raisebox{-3.5pt}{$\displaystyle #3$}%
}

\newcommand{\txtZQuot}[1]{\Z/#1\Z}

% just some shortcuts
\newcommand{\after}{\surround{\mskip4mu plus 2mu minus 1mu}{\mathord{\circ}}}
\newcommand{\bd}{\mathsf{b}}
\newcommand{\Cb}{C_{\bd}}
\newcommand{\defeq}{\coloneqq}
\newcommand{\EHb}{{E\!H}_{\bd}}
\newcommand{\eqdef}{\eqqcolon}
\newcommand{\Hb}{H_{\bd}}
\newcommand{\half}{\frac{1}{2}}
\newcommand{\hatCb}{\widehat{C}_{\bd}}
\newcommand{\homo}{^\mathsf{h}}
\newcommand{\isum}[1][1]{\sum_{i=#1}}
\newcommand{\ksum}[1][1]{\sum_{k=#1}}
\newcommand{\mr}{\mathrm}
\newcommand{\oper}{{}}
\newcommand{\scdot}{\,\cdot\,}
\newcommand{\setOneto}[1]{\{1,\ldots,#1\}}
\newcommand{\setZeroto}[1]{\{0,\ldots,#1\}}
\newcommand{\supr}{\infty}
%\newcommand{\supr}{{\mr{sup}}}
\newcommand{\surround}[2]{#1#2#1}
\newcommand{\thalf}{\tfrac{1}{2}}

% some text shortcuts
\qXq{iff}
\qXq{implies}
\qTXq{oder}
\qTXq{und}
\qqTXqq{und}
\newcommand{\fuer}[1][\qquad]{#1\text{für }}

%
\newcommand{\Achtung}{\emph{Achtung:} }

% xy tip selection (ComputerModern)
\SelectTips{cm}{}
\UseTips

% other xy specific settings
\newcommand{\xyhookdirspacing}{4pt}
\newdir{`}{\dir^{(}} 
\newdir{ `}{{}*!/-\xyhookdirspacing/\dir{`}}
\iffalse)\fi % fix syntax highlighting

% listing with -- is nicer than with bullets 
\setlist[itemize,1]{label=--,topsep=1pt,parsep=0pt,itemsep=0pt,leftmargin=2em}

%
\setkomafont{section}{\normalfont\large}
\pagestyle{empty}

%%%%%%%%%%%%%%%%%%%%%%%%%%%%%%%%%%%%%%%%%%%%%%%%%%%%%%%%%%%%%%%%%%%%%
%%% document
%%%%%%%%%%%%%%%%%%%%%%%%%%%%%%%%%%%%%%%%%%%%%%%%%%%%%%%%%%%%%%%%%%%%%

\begin{document}

%
\begin{tikzpicture}[remember picture,overlay]
    \node [yshift=-1.3cm,color=black!40] at (current page.north)
        {%
        \hfill%
        Seminar \enquote{Beschränkte Kohomologie} im
        SS~2014 an der Universität Regensburg%
        \hfill\mbox{}%
        };
\end{tikzpicture}

\vspace*{-0.5cm}
\begin{center}
    \Large Quasimorphismen
\end{center}

\medskip\noindent
J. Prem (\href{mailto:Johannes.Prem@stud.uni-regensburg.de}%
{\texttt{Johannes.Prem@stud.uni-regensburg.de}})
\hfill
06.~Mai 2014
\\[-8pt]
\rule{\textwidth}{0.4pt}

\smallskip\noindent
%
Ein \emph{Quasimorphismus} ist eine Abbildung, die
\enquote{bis auf einen beschränkten Fehler ein Gruppenhomomorphismen ist}.
In diesem Vortrag werden wir sehen, wie Quasimorphismen helfen können,
beschränkte Kohomologie von Gruppen zu berechnen.


\chapter{Vorwort, Notation und Vorbereitungen}
\section{Einleitung}
In den letzten Vorträgen haben wir \emph{beschränkte Kohomologie von Gruppen}
betrachtet und erste Eigenschaften selbiger gesehen. In diesem Skript werden
wir nun sogenannte \emph{Quasimorphismen} einführen und einen Zusammenhang zu
beschränkter Kohomologie beweisen \pcref{qmor:qmcoho}. Dabei verstehen wir unter
einem Quasimorphismus eine Abbildung, die \enquote{bis auf einen beschränkten
Fehler ein Gruppenhomomorphismus ist} \pcref{qmor:def}. Anschließend betrachten wir
gewisse \enquote{Gutartigkeitsbedingungen} von Quasimorphismen \pcref{homo}
und zeigen, wie man Quasimorphismen auf freien Produkten konstruieren kann
\pcref{freep}.

Im Folgenden betrachten wir die rellen Zahlen~$\R$ stets als Gruppe bezüglich
der üblichen Addition auf~$\R$ und sofern nicht anders beschrieben (oder
offensichtlich) schreiben wir die Verknüpfung auf einer Gruppe~$G$ immer
multiplikativ (wobei wir meist keine expliziten Verknüpfungszeichen setzen).

\section{Notation}
In diesem Skript wird folgende Notation verwendet:
\begin{itemize}
    \item
        Sowohl $\subset$ als auch $\subseteq$ stehen für: enthalten oder gleich.
        Echt enthalten wird durch $\subsetneq$ gekennzeichnet.
    
    \item
        Die \emph{natürlichen Zahlen $\N$} beginnen mit $0$.

    \item
        Ist $f\colon X\to\R$ eine Abbildung, so bezeichnet
        $\norm{f}_\supr = \sup_{x\in X}\,\abs{f(x)}$ die
        Supremumsnorm von~$f$.

    \item
        Für einen Ring~$R$ seien $\lCh R$ bzw. $\lCoCh R$
        die Kategorien der Kettenkomplexe bzw. Kokettenkomplexe, jeweils
        $\Z$-indiziert und in $\lMod R$, der Kategorie der Links-$R$-Moduln.
        Außerdem bezeichnet $\lGrad R$ die Kategorie der $\Z$-graduierten
        Links-$R$-Moduln.

    \item
        Sind $V$ und $W$ zwei normierte $\R$-Vektorräume, so ist
        $B(V,W)$ der bezüglich der Opertornorm normierte $\R$-Vektorraum
        aller stetig linearen Operatoren $V\to W$.
\end{itemize}

\section{Grundlegende Definitionen und Vorüberlegungen}
\label{ch3:basics}%
%
Sei $G$ eine Gruppe. Wir bezeichnen den \emph{Bar-Kettenkomplex} von~$G$
mit $C_\ast(G) = (C_\ast,\partial_\ast)$, wobei
\[ C_k(G) = \bigoplus_{g\in G^k} [g_1|\ldots|g_k]\,\Z \]
für alle $k\in\N$ und $C_k(G) = 0$ für $k\in\Z_{<0}$ sowie
\begin{align*}
    \partial_k\colon C_k(G) &\to C_{k-1}(G)
    \\
    [g_1|\ldots|g_k] &\mapsto [g_2|\ldots|g_k]
    + \isum^{k-1} (-1)^i \, [g_1|\ldots|g_ig_{i+1}|\ldots|g_k]
    + (-1)^k \, [g_1|\ldots|g_{k-1}]
\end{align*}
(auf Erzeugern) für alle $k\in\N$. Der \emph{Bar-Kokettenkomplex mit reellen
Koeffizienten} ist der (algebraisch) dualisierte Kokettenkomplex
\[ C^\ast(G;\R) = \Hom_\Z\bigl( C_\ast(G), \R \bigr)  \;\in\lCoCh{\R} , \]
wobei wir die Folge der induzierten Korandoperatoren mit~$\delta$ bezeichnen.
Der \emph{reelle Bar-Kettenkomplex}
\[ C_\ast^{\R}(G)
    \defeq C_\ast(G) \underset{\scriptscriptstyle\Z}\otimes \R
    \;\in\lCh{\R}
\]
wird zusammen mit der $\ell^1$-Norm bezüglich der Basis zu einem
normierten Kettenkomplex. Der topologisch dualisierte Kokettenkomplex
\[ \Cb^\ast(G;\R) \defeq B\bigl( C_\ast^{\R}(G), \R \bigr) \]
ist der \emph{Banach-Bar-Kokettenkomplex} zu~$G$ und
besteht aus Banachräumen bezüglich der Operatornorm und der Folge der
induzierten Korandoperatoren~$\delta_\bd$.

Die \emph{(gewöhnliche) (Gruppen-)Kohomologie} von $G$ (mit reellen
Koeffizienten) ist
\[ H^\ast(G;\R) = H^\ast\bigl( C^\ast(G;\R) \bigr)  , \]
wobei $H^\ast\colon\lCh{\R}\to\lGrad{\R}$ den algebraischen Kohomologiefunktor
bezeichnet. Die \emph{beschränkte (Gruppen-)Kohomologie} von $G$ ist
\[ \Hb^\ast(G;\R) = H^\ast\bigl( \Cb^\ast(G;\R) \bigr)  . \]

Nach Definition gilt $C_1(G) = \bigoplus_{g_1\in G}\, [g_1]\,\Z$. Fassen wir
$G$ vermöge der Inklusion auf die entsprechenden Basiselemente aus $C_1(G)$
als Teilmenge von $C_1(G)$ auf, so ist ein Kokettenelement $f\in C^1(G;\R)$ also 
eindeutig durch die Werte auf $G$ bestimmt. Dies liefert einen
$\R$-Vektorraum-Isomorphismus 
\[ C^1(G;\R) \cong \R^G  \]
zwischen $C^1(G;\R)$ und dem Raum aller $\R$-wertigen Funktionen auf~$G$.
Wir werden daher (etwas schlampig) im Folgenden nicht zwischen den beiden
$\R$-Vektorräumen unterscheiden und $C^1(G;\R)$ jeweils dem Kontext entsprechend
interpretieren.

Ähnlich gehen wir bei $\Cb^1(G;\R)$ vor: Ein Element $f\in\Cb^1(G;\R)$ ist ein
stetig lineares Funktional $\bigoplus_{g_1\in G}\, [g_1]\,\R = C_1^{\R} \to \R$.
Somit ist auch hier $f$ eindeutig durch die Werte auf~$G$ bestimmt, aber
zusätzlich muss $\norm{f}_\oper < \infty$ für die Operatornorm von~$f$ gelten.
Bezeichnet $f_0\colon G\to\R$ die zu $f$ assoziierte Funktion auf~$G$, so
überlegt man sich leicht, dass
\[ \norm{f}_\oper = \norm{f_0}_\supr \]
gilt. Die Bedingung, dass $f$ stetig ist, ist also äquivalent dazu, dass
$f_0$ beschränkt ist. Damit erhalten wir einen isometrischen Isomorphismus
zwischen $\Cb^1(G;\R)$ und dem $\R$-Vektorraum aller beschränkten Funktionen
auf~$G$ zusammen mit der Supremumsnorm. Wir unterscheiden daher im Folgenden
(auch wieder etwas schlampig) nicht zwischen den beiden normierten Räumen und
interpretieren $\Cb^1(G;\R)$ dem Kontext entsprechend.

Analog entsprechen beschränkte Funktionen $G^k\to\R$ für $k\in\N$ den Elementen
in $\Cb^k(G;\R)$ und man zeigt auch allgemein leicht, dass für $f\in\Cb^k(G;\R)$
und die entsprechende Funktion $f_0\colon G^k\to\R$ stets
$\norm{f}_\oper = \norm{f_0}_\supr$ gilt.

\section{Quasimorphismen und Zusammenhang zu beschränkter Kohomologie}
\begin{thDef}[Defekt, Quasimorphismus, trivialer Quasimorphismus] \hfill
    \begin{itemize}
        \item
            Der \emph{Defekt~$D(f)$} einer Abbildung $f\colon G\to\R$
            ist
            \[ D(f) \defeq \sup_{g,h\in G}\, \abs{f(g) + f(h) - f(gh)}
                \;\in\, \R[\geq0] \cup \{\infty\}
            . \]
            
        \item
            Ein \emph{Quasimorphismus auf $G$} ist eine
            Abbildung $f\colon G\to\R$ mit $D(f) < \infty$.
            
        \item
            Ein Quasimorphismus $f$ auf~$G$ ist ein
            \emph{trivialer Quasimorphismus}, wenn es einen
            Gruppenhomomorphismus~$f'\colon G\to\R$ mit
            $\norm{f-f'}_\supr < \infty$ gibt.
            
        \item
            Wir schreiben $\QM(G)$ (bzw. $\QM_0(G)$) für den $\R$-Vektorraum
            aller Quasimorphismen (bzw. aller trivialen Quasimorphismen). 
    \end{itemize}
\end{thDef}

\begin{thProposition}[Zerlegung von \texorpdfstring{$\QM_0$}{QM0}]
    \label{qmor:decompQM0}%
    %
    Es gilt: $\displaystyle\QM_0(G) = \Hom(G,\R) \oplus \Cb^1(G;\R)$.
\end{thProposition}

\begin{thAufgabe}
    Beweise \cref{qmor:decompQM0}.
\end{thAufgabe}

\begin{thErinnerDef}[Vergleichsabbildung, $\EHb^\ast$] \hfill
    \begin{itemize}
        \item
            Die Inklusion von Kokettenkomplexen $\Cb^\ast(G;\R) \hookrightarrow
            C^\ast(G;\R)$ induziert eine Abbildung $\Hb^\ast(G;\R) \to
            H^\ast(G;\R)$ in Kohomologie, die sogenannte
            \emph{Vergleichsabbildung}.
        \item
            Der (gradweise) Kern der Vergleichsabbildung ist
            \[ \EHb^\ast(G;\R)
                \defeq \ker\bigl( \Hb^\ast(G;\R) \to H^\ast(G;\R) \bigr)
                \;\in\, \lGrad{\R}
            . \]
    \end{itemize}
\end{thErinnerDef}

\begin{thSatz}[Zusammenhang zwischen Quasimorphismen und Kohomologie]
    \label{qmor:qmcoho}%
    %
    Es gilt:
    \[ \Quot{\QM(G)}{\QM_0(G)} \;\cong\; \EHb^2(G;\R)  . \]
\end{thSatz}

\enlargethispage{1cm}
\begin{proofsketch}
    Man kann zeigen, dass es lineare Abbildungen $\delta$ und $\phi$ gibt, die
    das folgende Diagramm kommutativ machen:
    \[
        \newcommand{\gray}[1]{{\color{black!50}#1}}
        \newcommand{\graylap}[1]{\mathrlap{\gray{#1}}}
        \xymatrix@C=1cm@R=0.4cm{
            C^1(G;\R) \ar[r]^(.55){\delta^1}
            & \ker(\delta^2) \ar[r]^{\pi}
            & H^2(G;\R) \ar@{}[r]|-{\gray=}
            & \graylap{\ker\bigl(\delta^2\bigr)/\im\bigl(\delta^1\bigr)}
            \\
            \QM(G) \ar@{-->}[r]^(.55){\delta^{\phantom{1}}} \ar@{ `->}[u]
                \ar@{-->}[rrd]^(.68)\phi
            & \ker(\delta_\bd^2) \ar[r]^{\pi_\bd} \ar@{ `->}[u]
            & \Hb^2(G;\R) \ar@{}[r]|-{\gray=} \ar[u]_c
            & \graylap{\ker\bigl(\delta^2_\bd\bigr)/\im\bigl(\delta^1_\bd\bigr)}
            \\
            & & \EHb^2(G;\R) \ar@{}[r]|-{\gray=} \ar@{ `->}[u]
                & \graylap{\ker(c)}
            %
            & \hspace*{1.5cm} % alignment hack :/
        }
    \]
    (wobei $c$ die Vergleichsabbildung ist, $\pi$ und $\pi_\bd$ die kanonischen
    Projektionen sind und die linke Inklusion im Sinne von \cref{setup} zu
    verstehen ist). Durch eine genaue Untersuchung des Kerns und Bilds
    von~$\phi$ erhält man die Behauptung aus dem Isomorphiesatz.
    \\
\end{proofsketch}

\chapter{Ungerade und homogene Quasimorphismen}
\begin{thSetup}
    Sei $G$ auch in diesem Kapitel eine Gruppe.
\end{thSetup}

\begin{thDef}[Ungerader/Homogener Quasimorphismus]
    Sei $f\in\QM(G)$ ein Quasimorphismus.
    \begin{itemize}
        \item
            Wir sagen, $f$ ist \emph{ungerade}, falls
            \[ \forall\,g\in G\colon\quad
                f(g^{-1}) = -f(g)
            \]
            gilt.
            
        \item
            Wir sagen, $f$ ist \emph{homogen}, falls
            \[ \forall\,g\in G,\;k\in\Z\colon\quad
                f(g^k) = k\,f(g)
            \]
            gilt.
    \end{itemize}
\end{thDef}

\begin{thProposition}[Homogenisierung]
    Sei $f\in\QM(G)$. Dann ist die Abbildung
    \[ f\homo \colon G \to \R, \quad
        x\mapsto \lim_{n\to\infty} \frac{f(x^n)}{n}
    \]
    wohldefiniert und ein homogener Quasimorphismus, welchen wir
    \emph{Homogenisierung von $f$} nennen. Außerdem gilt
    \[ \norm{f-f\homo}_\supr \leq D(f)  . \]
\end{thProposition}

Zum Beweis verwenden wir das folgende Lemma:

\begin{thLemma}
    Sei $f\in\QM(G)$, sei $n\in\N$ und seien
    $w_1,\dots,w_n\in G$. Dann gilt für $w\defeq w_1\cdots w_n\in G$:
    \[ \abs[\Big]{ f(w) - \isum^n f(w_i) } \leq (n-1) D(f)  . \]
\end{thLemma}

\begin{thKorollar}
    Es gilt
    \[ \QM(G) = \QM\homo(G) \oplus \Cb^1(G)  , \]
    wobei $\QM\homo(G) \subset \QM(G)$ den Untervekorraum aller homogenen
    Quasimorphismen bezeichnet. Insbesondere gilt auch
    \[ \Quot{\QM\homo(G)}{\Hom_\Z(G,\R)} \;\cong\; E\Hb^2(G;\R)  . \]
\end{thKorollar}

\begin{thKorollar}[Quasimorphismen auf abelschen Gruppen]
    Sei $G$ abelsch. Dann gilt gibt es keine nicht-trivialen
    Quasimorphismen auf~$G$, d.\,h. es gilt $\QM(G) = \QM_0(G)$.
\end{thKorollar}

\chapter{Quasimorphismen auf freien Produkten} \label{freep}
In diesem Kapitel wollen wir eine Konstruktion angeben, die (eine potentiell
große Anzahl, vgl. \cref{freep:cohofreegrp}) von Quasimorphismen auf freien
Produkten und insbesondere freien Gruppen liefert. Der Großteil dieses Kapitels
basiert auf Rolli\cite{arxiv:rolli09}.

\begin{thKonstruktionSetup}
    \label{freep:construction}
    %
    Sei $I$ eine Menge mit $\abs{I}\geq 2$ und $(G_i)_{i\in I}$ eine Familie
    nicht-trivialer Gruppen. Sei weiter
    \[ G \defeq \freeprod_{i\in I} G_i \]
    das freie Produkt der Gruppen. Wir behandeln jede der Gruppen~$G_i$
    unter der üblichen Identifikation als Untergruppe von~$G$ und bezeichnen
    mit $()\in G$ das leere Wort.
    Ist nun $x\in G\setminus\{()\}$, so gibt es ein $n\in\N$ und Indizes
    $i_1,\dots,i_n\in I$, so dass $x$ eine eindeutige Darstellung
    \[ x = x_1 \cdots x_n \]
    besitzt, wobei folgende Bedingungen gelten: Für alle $k\in\setOneto n$ ist
    $x_k \in G_{i_k}$ nicht-trivial und für alle $k\in\setOneto{n-1}$ gilt
    $i_k\neq i_{k+1}$.
    
    Wir möchten nun Quasimorphismen auf~$G$ konstruieren. Für $i\in I$ sei dazu
    $\hatCb^1(G_i;\R)$ derjenige Unterraum von $\Cb^1(G_i;\R)$, der alle ungeraden
    beschränkten Quasimorphismen enthält. Weiter sei
    \[ V(G) \defeq \prod_{i\in I} \hatCb^1(G_i;\R) \]
    und
    \[ V_0(G) \defeq \bigl\{ (\sigma_i)_{i\in I} \in V(G) \Mid
        \sup\nolimits_{i\in I}\, \norm{\sigma_i}_\supr < \infty \bigr\}
    . \]%
    \rule{0pt}{1.3\ht\strutbox}%
    Für $\sigma = (\sigma_i)_{i\in I} \in V_0(G)$ definieren wir nun eine
    Abbildung~$g_\sigma$ wie folgt:
    \begin{align*}
        g_\sigma\colon G &\to \R
        \\
        x &\mapsto 
        \begin{cases}
            0, \quad& \text{falls $x=()$},
            \\
            \ksum^n \sigma_{i_k}(x_k), \quad&
            \text{falls $x = x_1\cdots x_n$ im obigen Sinne} .
        \end{cases}
    \end{align*}
\end{thKonstruktionSetup}

\begin{thProposition}[Quasimorphismen auf freien Produkten]
    \label{freep:qmonfreep}%
    %
    Die Abbildung $g_\sigma$ aus \cref{freep:construction} ist für
    alle $\sigma\in V_0(G)$ ein Quasimorphismus.
\end{thProposition}

\begin{proof}
    Seien $x,y\in G$ mit Darstellungen $x = x_1\cdots x_n$ bzw.
    $y = y_1\cdots y_m$ und zugehörigen Indizes $i_1,\ldots,i_n$ bzw.
    $j_1,\ldots,j_m$ wie in \cref{freep:construction}. Gilt nun $i_n \neq j_1$,
    so hat $xy\in G$ einfach die Darstellung $x_1\cdots x_n y_1\cdots y_m$.
    Gilt andererseits $i_n = j_1$, so ist $z \defeq x_ny_1$ entweder
    nicht-trivial in $G_{i_n} = G_{j_1}$ und damit hat $xy$ die Darstellung
    \[ x_1\cdots x_{n-1} \, z \, y_2\cdots y_m  , \]
    oder aber $z$ ist das neutrale Element in $G_{i_n}$, womit $x_n$ gegen
    $y_1$ wegfällt und wir $x_{n-1} y_2$ nach demselben Prinzip betrachten
    müssen. Da beide Darstellungen endlich sind, muss dieser Prozess aber
    irgendwann aufhören und wir erhalten für $xy$ eine Darstellung
    \[ x_1\cdots x_{n-r} \, z \, y_{r+1}\cdots y_m  , \]
    mit $r\in\N$ und $z\in G_{i_{n-r+1}} = G_{j_r}$ oder aber $z=()$.
    In jedem Fall gilt aber $x_{n-k} = y_{k+1}^{-1}$ für alle
    $k\in\setZeroto{r-2}$, und weil $\sigma$ nach Konstruktion aus ungeraden
    Quasimorphismen besteht, gilt
    \[ \sigma_{i_{n-k}}(x_{n-k}) + \sigma_{j_{k+1}}(y_{k+1}^{-1}) = 0 \]
    für alle $k\in\setZeroto{r-2}$. Damit rechnen wir:
    \begin{align*}
        \abs[\big]{ g_\sigma(x) + g_\sigma(y) - g_\sigma(xy) }
        &= \abs[\Big]{
                \ksum^n \sigma_{i_k}(x_k) + \ksum^m \sigma_{j_k}(y_k)
                - \ksum^{n-r} \sigma_{i_k}(x_k)
                - \sigma_{j_r}(z)
                - \ksum[r+1]^m \sigma_{j_k}(y_k)
            }
        \\[0.5ex]
        &= \abs[\big]{
            \sigma_{i_{n-r+1}}(x_{n-r+1})
            + \sigma_{j_r}(y_r)
            - \sigma_{j_r}(z)
            }
        \\[0.5ex]
        &\leq \norm{\sigma_{i_{n-r+1}}}_\supr
            + \norm{\sigma_{j_r}}_\supr
            + \norm{\sigma_{j_r}}_\supr
        \\[0.5ex]
        &\leq 3\cdot\sup_{i\in I}\, \norm{\sigma_i}_\supr
    \end{align*}
    Es folgt
    \[ D(g_\sigma) \leq 3\cdot\sup_{i\in I}\, \norm{\sigma_i}_\supr < \infty \]
    und damit ist $g_\sigma$ wie behauptet ein Quasimorphismus auf~$G$.
    \\
\end{proof}

Offenbar ist $V_0(G)$ ein Untervektorraum von $V(G)$, womit folgende Behauptung
Sinn ergibt:

\begin{thProposition}
    \label{freep:qmtocoho}%
    %
    Die Abbildung
    \[ V_0(G) \to \EHb^2(G;\R), \quad
        \sigma \mapsto \bigl[\delta^1(g_\sigma)\bigr]_\bd
    \]
    (mit $g_\sigma$ aus \cref{freep:construction})
    ist wohldefiniert, linear und injektiv.
\end{thProposition}

\begin{proof}
    Wie im Beweis von \cref{qmor:qmcoho} sieht man, dass die Abbildung
    wohldefiniert ist und die Linearität ist klar nach Konstruktion.
    Sei $\sigma\in V_0(G)$ mit
    $[\delta^1(g_\sigma)]_\bd = 0 \in\Hb^2(G;\R)$. Daraus folgt aber
    (wie im Beweis von \cref{qmor:qmcoho}), dass $g_\sigma - b = f$
    für ein $b\in\Cb^1(G;\R)$ und ein $f\in\Hom_\Z(G,\R)$ gilt. Seien
    $i\in I$, $x\in G_i$ und $k\in\N$. Dann gilt
    \[ g_\sigma(x^k) - b(x^k) = \sigma_i(x^k) - b(x^k) 
        \leq \norm{\sigma_i}_\supr + \norm{b}_\supr
    , \]
    womit
    \[ f(x) \leq \frac{\norm{\sigma_i}_\supr + \norm{b}_\supr}{k}
        \;\xrightarrow[k\to\infty]{}\; 0
    , \]
    also $f(x) = 0$ folgt. Weil $\bigcup_{i\in I} G_i \subset G$ ein
    Erzeugendensystem von $G$ ist, gilt somit schon $f=0$, d.\,h.
    $g_\sigma = b \in\Cb^1(G;\R)$. Seien nun $i,j\in I$ mit $i\neq j$ und
    seien $x\in G_i,\; y\in G_j$. Für alle $k\in\Z$ gilt dann
    \[ g_\sigma\bigl( (xy^{\pm1})^k \bigr)
        = k\cdot \bigl( \sigma_i(x) \pm \sigma_j(y) \bigr)
    . \]
    Weil $g_\sigma = b$ beschränkt ist, folgt 
    \[ \sigma_i(x) \pm \sigma_j(y) = 0
        \qtextq{und daraus}
        \sigma_i(x) = 0 = \sigma_j(y)
    . \]
    Da $i,j,x,y$ beliebig waren, folgt $\sigma = 0$. Damit ergibt sich
    die Injektivität der Abbildung aus der Behauptung.
    \\
\end{proof}

\begin{thKorollar}[Beschränkte Kohomologie freier Gruppen ist nicht trivial]
    \label{freep:cohofreegrp}%
    %
    Sei $F$ eine freie Gruppe vom Rang mindestens~$2$. Dann gilt
    \[ \dim_{\R} \Hb^2(F;\R) = \infty  . \]
\end{thKorollar}

\begin{proof}
    Sei $S\subset F$ ein freies Erzeugendensystem von~$F$. Dann gilt
    $F \cong \freeprod_S \Z$, womit es nach \cref{freep:qmtocoho} genügt,
    zu zeigen, dass der Raum $\hatCb^1(\Z;\R)$ unendlich-dimensional ist. Man
    sieht aber leicht, dass
    \begin{align*}
        \ell^\infty(\R) &\to \hatCb^1(\Z;\R)
        \\
        (x_n)_{n\in\N} &\mapsto \left( 
        \begin{aligned}
            \Z &\to \R \\
             k &\mapsto \sgn(k) \, x_{\abs{k}}
        \end{aligned}
        \,\right)
    \end{align*}
    ein Isomorphismus zwischen dem unendlich-dimensionalen $\R$-Vektorraum der
    reellwertigen beschränkten Folgen und $\hatCb^1(\Z;\R)$ ist. (Dabei ist
    \enquote{$\sgn$} die Signumsfunktion.)
    \\
\end{proof}

\begin{thBeispiel}%
    [Beschränkte Kohomologie von $\operatorname{PSL}_2(\Z)$ im Grad~2]
    %
    Wir betrachten die Gruppe $\operatorname{PSL}_2(\Z) \cong \txtZQuot2 \ast
    \txtZQuot3$. Um zu zeigen, dass $\Hb^2( \operatorname{PSL}_2(\Z); \R )$
    nicht trivial ist, genügt es nach \cref{freep:qmtocoho} zu zeigen, dass 
    \[ V_0\bigl(\operatorname{PSL}_2(\Z)\bigr)
        \cong \hatCb^1(\txtZQuot2;\R) \times \hatCb^1(\txtZQuot3;\R)
    \]
    nicht trivial ist. Sei $n\in\N$. Ein ungerader (notwendigerweise
    beschränkter) Quasimorphismus auf $\txtZQuot n$ ist offensichtlich
    durch die Werte auf den Restklassen von $1$ bis $\lfloor (n-1)/2 \rfloor$
    eindeutig bestimmt. Für $\txtZQuot2$ gilt also $\hatCb^1(\txtZQuot2;\R) =
    \{0\}$. Aber für $\txtZQuot3$ können wir den Wert auf der Restklasse von~$1$
    frei vorgeben, so dass wir 
    \[ \hatCb^1(\txtZQuot3;\R) \cong \R \qtextq{und damit}
        V_0\bigl(\operatorname{PSL}_2(\Z)\bigr) \cong \R
    \]
    erhalten.
\end{thBeispiel}


\enlargethispage{2cm}
\small
\nocite{bookc:calegari09}
\nocite{lecnotes:loeh:bdcoho}
\bibliographystyle{plaindin}
\bibliography{bibsources}

\end{document}





