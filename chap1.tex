\chapter{Quasimorphismen und Zusammenhang zu beschränkter Kohomologie}
\begin{thSetup}
    Sei $G$ in diesem Kapitel stets eine Gruppe.
\end{thSetup}

\begin{thDef}[Defekt, Quasimorphismus] \label{qmor:def} \hfill
    \begin{itemize}
        \item
            Der \emph{Defekt~$D(f)$} einer Abbildung $f\colon G\to\R$
            ist
            \[ D(f) \defeq \sup_{g,h\in G}\, \abs{f(g) + f(h) - f(gh)}
                \;\in\, \R[\geq0] \cup \{\infty\}
            . \]
            
        \item
            Ein \emph{Quasimorphismus auf $G$} ist eine
            Abbildung $f\colon G\to\R$ mit $D(f) < \infty$.
    \end{itemize}
\end{thDef}

\begin{BspList}
\item
    Ein Quasimorphismus $G\to\R$ hat genau dann Defekt~$0$, wenn er ein
    Gruppenhomomorphismus $G\to\R$ ist.

\item
    Jede beschränkte Abbildung $f\colon G\to\R$ ist ein
    Quasimorphismus mit $D(f) \leq 3\norm{f}_\supr$.

\item
    Allgemeiner ist für jeden Gruppenhomomorphismus $f\colon G\to\R$ und
    jede beschränkte Abbildung $b\colon G\to\R$ die (punktweise) Summe $f+b$
    ein Quasimorphismus.
\end{BspList}

Das letzte Beispiel gibt Anlass zu folgender Definition:

\begin{thDef}[Trivialer Quasimorphismus]
    Sei $f\colon G\to\R$ ein Quasimorphismus. Dann ist $f$ ein \emph{trivialer
    Quasimorphismus}, falls es einen Gruppenhomomorphismus $f'\colon G\to\R$
    mit $\norm{f-f'}_\supr < \infty$ gibt.
\end{thDef}

Offensichtlich bildet die Menge aller Quasimorphismen (bzw. aller trivialen
Quasimorphismen) einen $\R$-Vektorraum.

\begin{thDef}[Vektorraum aller (trivialen) Quasimorphismen]
    Wir schreiben $\QM(G)$ (bzw. $\QM_0(G)$) für den $\R$-Vektorraum aller
    Quasimorphismen (bzw. aller trivialen Quasimorphismen). 
\end{thDef}

\begin{thProposition}[Zerlegung von \texorpdfstring{$\QM_0$}{QM0}]
    \label{qmor:decompQM0}%
    %
    Es gilt:
    \[ \QM_0(G) = \Hom(G,\R) \oplus \Cb^1(G;\R)  . \]
\end{thProposition}

\begin{proofsketch}
    Die Gleichheit $\QM_0(G) = \Hom(G,\R) + \Cb^1(G;\R)$ liest man direkt an den
    Definitionen ab und weiter überlegt man sich leicht, dass es keine beschränkten
    Gruppenhomomorphismen $G\to\R$ gibt.
    \\
\end{proofsketch}

\begin{thBeispiel}[Wörterzählende Quasimorphismen]
    Nicht-triviale Beispiele liefern sogenannte \enquote{counting
    quasimorphisms}, welche wir im Folgenden auf Deutsch als
    \emph{wörterzählende Quasimorphismen} bezeichnen. Sei $F$ die freie Gruppe
    über einer Menge~$S$. Elemente aus $F$ seien stets reduzierte Wörter in den
    Erzeugern aus~$S$. Ist nun $w\in F$, so kann man zeigen, dass die Abbildung
    \begin{align*}
        \psi_w\colon F &\to \R  \\
        g &\mapsto (\text{Anzahl der Vorkommen von $w$ in $g$})
                 - (\text{Anzahl der Vorkommen von $w^{-1}$ in $g$})
    \end{align*}
    einen Quasimorphismus auf~$F$ liefert. Einen Beweis dieser Behauptung findet
    man beispielsweise bei
    Löh\cite[Ch.\,2,\;2.5.3,\;Lemma~2.5.11]{lecnotes:loeh:bdcoho}.
    Für eine ausführlichere Behandlung (auch anderer Varianten) wörterzählender
    Quasimorphismen siehe Caligari\cite[Ch.\,2,\;2.3.2]{bookc:calegari09}.
\end{thBeispiel}

\begin{thErinnerDef}[Vergleichsabbildung, $\EHb^\ast$] \hfill
    \begin{itemize}
        \item
            Die Inklusion von Kokettenkomplexen $\Cb^\ast(G;\R) \hookrightarrow
            C^\ast(G;\R)$ induziert eine Abbildung $\Hb^\ast(G;\R) \to
            H^\ast(G;\R)$ in Kohomologie, die sogenannte
            \emph{Vergleichsabbildung}.
        \item
            Es ist
            \[ \EHb^\ast(G;\R)
                \defeq \ker\bigl( \Hb^\ast(G;\R) \to H^\ast(G;\R) \bigr)
                \;\in\, \lGrad{\R}
            \]
            der (gradweise) Kern der Vergleichsabbildung.
    \end{itemize}
\end{thErinnerDef}

\begin{thSatz}[Zusammenhang zwischen Quasimorphismen und Kohomologie]
    \label{qmor:qmcoho}%
    %
    Es gilt:
    \[ \Quot{\QM(G)}{\QM_0(G)} \;\cong\; \EHb^2(G;\R)  . \]
\end{thSatz}

\begin{proof}
    Wir betrachten das folgende Diagramm:
    \[
        \newcommand{\gray}[1]{{\color{black!50}#1}}
        \newcommand{\graylap}[1]{\mathrlap{\gray{#1}}}
        \xymatrix@C=1cm{
            C^1(G;\R) \ar[r]^{\delta^1}
            & C^2(G;\R) \ar[r]^{\pi}
            & H^2(G;\R) \ar@{}[r]|-{\gray=}
                & \graylap{\ker\bigl(\delta^2\bigr)/\im\bigl(\delta^1\bigr)}
            \\
            \QM(G;\R) \ar@{-->}[r]^{\delta} \ar@{ `->}[u]
                \ar@{-->}[rrd]^(.6)\phi
                & \Cb^2(G;\R) \ar[r]^{\pi_\bd} \ar@{ `->}[u]
            & \Hb^2(G;\R) \ar@{}[r]|-{\gray=} \ar[u]_c
                & \graylap{\ker\bigl(\delta^2_\bd\bigr)/\im\bigl(\delta^1_\bd\bigr)}
            \\
            & & \EHb^2(G;\R) \ar@{}[r]|-{\gray=} \ar@{ `->}[u]
                & \graylap{\ker(c)}
            %
            & \hspace*{1.5cm} % alignment hack :/
        }
    \]
    (wobei $c$ die Vergleichsabbildung ist, $\pi$ und $\pi_\bd$ die
    kanonischen Projektionen sind und die linke Inklusion im Sinne
    von \cref{ch3:basics} zu verstehen ist). Sei $f\in\QM(G;\R)$.
    Dann gilt $\delta^1(f) = f\after\partial_2$ und nach Definition von
    $\partial_2$ somit
    \[ (f\after\partial_2)\bigl([g_1|g_2]\bigr) 
        = f(g_2) - f(g_1g_2) + f(g_1)
    \]
    für alle $g_1,g_2\in G$. Nach den Vorüberlegungen in \cref{ch3:basics} gilt
    damit
    \[ \norm[\big]{\delta^1(f)}
        \, = \sup_{g_1,g_2\in G} \abs{f(g_2) - f(g_1g_2) + f(g_1)}
        \,=\, D(f) \,<\, \infty
    . \]
    Also faktorisiert $\QM(G;\R) \longhookrightarrow C^1(G;\R)
    \overset{\smash{\delta^1}}\longto C^2(G;\R)$ über $\Cb^2(G;\R)$ und
    wir bezeichnen die induzierte Abbildung einfach mit~$\delta$.
    Weil dann auch das äußere Rechteck im obigen Diagramm kommutiert,
    gilt $c\after\pi_\bd\after\delta = 0$. Nach der universellen Eigenschaft
    des Kerns erhalten wir somit eine Abbildung~$\phi$ wie im Diagramm,
    so dass auch das untere Dreieck kommutiert.
    Sei nun $[g]_\bd\in\EHb^2(G;\R)$, d.\,h. mit $[g] = 0 \in H^2(G;\R)$.
    Dann gilt:
    \[ [g] = 0
        \implies g\in\im\bigl(\delta^1\bigr)
        \implies \exists\,\tilde g\in C^1(G;\R)\colon
            \delta^1(\tilde g) = g \in \Cb^2(G;\R)
    . \]
    Solch ein $\tilde g$ ist dann offenbar ein Quasimorphismus. Es folgt die
    Surjektivität von~$\phi$. Sei $f\in\ker(\phi)$, also
    \[ \phi(f) = 0 \in\ker(c) \subset \Hb^2(G;\R)  \qtextq{bzw.}
        \delta(f) \in \im\bigl(\delta^1_\bd\bigr)
    . \]
    Dann gibt es ein $\tilde f\in\Cb^1(G;\R)$ mit $\delta^1_\bd(\tilde f)
    = \delta(f)$ bzw. $\delta^1(f - \tilde f) = 0 \in\Cb^2(G;\R)$.
    Es folgt $f - \tilde f \in\Hom(G,\R)$ und damit
    \[ f \in \Hom(G,\R) + \Cb^1(G;\R) = \QM_0(G)  . \]
    Dies zeigt $\ker(\phi) \subset \QM_0(G)$. Sei umgekehrt
    $f = f'+b \in\QM_0(G)$ mit $f'\in\Hom(G,\R)$ und $b\in\Cb^1(G;\R)$.
    Dann gilt
    \[ \bigl[\delta(f'+b)\bigr]_\bd 
        = \bigl[0 + \delta(b)\bigr]_\bd = 0 \in \Hb^2(G;\R)
    ; \]
    also folgt $\QM_0(G) \subset \ker(\phi)$ und damit Gleichheit. Der
    Homomorphiesatz liefert nun die Behauptung.
    \\
\end{proof}

\begin{thKorollar}
    Verschwindet die zweite Kohomologie $H^2(G;\R)$ von $G$, so gilt
    \[ \Quot{\QM(G)}{\QM_0(G)} \;\cong\; \Hb^2(G;\R)  . \]
\end{thKorollar}

\begin{thBeispiel}[Quasimorphismen auf freien Gruppen]
    Sei $F$ eine freie Gruppe vom Rang mindestens~$2$. Aus dem vorherigen
    Vortrag wissen wir, % TODO: Ist das so?
    dass $\dim_{\R} \Hb^2(F;\R) = \infty$ gilt (oder siehe
    \cref{freep:cohofreegrp}). Außerdem kann man zeigen, dass $H^2(F;\R)$
    verschwindet (siehe zum Beispiel
    Löh\cite[Ch.\,1,\;1.3.4,\;Example~1.3.13]{lecnotes:loeh:bdcoho}).
    Dann folgt aus dem Korollar sofort
    \[ \dim_{\R} \Quot{\QM(F)}{\QM_0(F)} = \infty  , \]
    d.\,h. es gibt unendlich viele (linear unabhängige) nicht-triviale
    Quasimorphismen auf~$F$.
\end{thBeispiel}
