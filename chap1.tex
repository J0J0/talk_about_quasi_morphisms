\section{Quasimorphismen und Zusammenhang zu beschränkter Kohomologie}
\begin{thDef}[Defekt, Quasimorphismus, trivialer Quasimorphismus] \hfill
    \begin{itemize}
        \item
            Der \emph{Defekt~$D(f)$} einer Abbildung $f\colon G\to\R$
            ist
            \[ D(f) \defeq \sup_{g,h\in G}\, \abs{f(g) + f(h) - f(gh)}
                \;\in\, \R[\geq0] \cup \{\infty\}
            . \]
            
        \item
            Ein \emph{Quasimorphismus auf $G$} ist eine
            Abbildung $f\colon G\to\R$ mit $D(f) < \infty$.
            
        \item
            Ein Quasimorphismus $f$ auf~$G$ ist ein
            \emph{trivialer Quasimorphismus}, wenn es einen
            Gruppenhomomorphismus~$f'\colon G\to\R$ mit
            $\norm{f-f'}_\supr < \infty$ gibt.
            
        \item
            Wir schreiben $\QM(G)$ (bzw. $\QM_0(G)$) für den $\R$-Vektorraum
            aller Quasimorphismen (bzw. aller trivialen Quasimorphismen). 
    \end{itemize}
\end{thDef}

\begin{thProposition}[Zerlegung von \texorpdfstring{$\QM_0$}{QM0}]
    \label{qmor:decompQM0}%
    %
    Es gilt: $\displaystyle\QM_0(G) = \Hom_\Z(G,\R) \oplus \Cb^1(G;\R)$.
\end{thProposition}

\begin{thAufgabe}
    Beweise \cref{qmor:decompQM0}.
\end{thAufgabe}

\begin{thErinnerDef}[Vergleichsabbildung, $\EHb^\ast$] \hfill
    \begin{itemize}
        \item
            Die Inklusion von Kokettenkomplexen $\Cb^\ast(G;\R) \hookrightarrow
            C^\ast(G;\R)$ induziert eine Abbildung $\Hb^\ast(G;\R) \to
            H^\ast(G;\R)$ in Kohomologie, die sogenannte
            \emph{Vergleichsabbildung}.
        \item
            Der (gradweise) Kern der Vergleichsabbildung ist
            \[ \EHb^\ast(G;\R)
                \defeq \ker\bigl( \Hb^\ast(G;\R) \to H^\ast(G;\R) \bigr)
                \;\in\, \lGrad{\R}
            . \]
    \end{itemize}
\end{thErinnerDef}

\begin{thSatz}[Zusammenhang zwischen Quasimorphismen und Kohomologie]
    \label{qmor:qmcoho}%
    %
    Es gilt:
    \[ \Quot{\QM(G)}{\QM_0(G)} \;\cong\; \EHb^2(G;\R)  . \]
\end{thSatz}

\enlargethispage{1cm}
\begin{proofsketch}
    Man kann zeigen, dass es lineare Abbildungen $\delta$ und $\phi$ gibt, die
    das folgende Diagramm kommutativ machen:
    \[
        \newcommand{\gray}[1]{{\color{black!50}#1}}
        \newcommand{\graylap}[1]{\mathrlap{\gray{#1}}}
        \xymatrix@C=1cm@R=0.4cm{
            C^1(G;\R) \ar[r]^{\delta^1}
            & C^2(G;\R) \ar[r]^{\pi}
            & H^2(G;\R) \ar@{}[r]|-{\gray=}
                & \graylap{\ker\bigl(\delta^2\bigr)/\im\bigl(\delta^1\bigr)}
            \\
            \QM(G;\R) \ar@{-->}[r]^{\delta} \ar@{ `->}[u]
                \ar@{-->}[rrd]^(.68)\phi
                & \Cb^2(G;\R) \ar[r]^{\pi_\bd} \ar@{ `->}[u]
            & \Hb^2(G;\R) \ar@{}[r]|-{\gray=} \ar[u]_c
                & \graylap{\ker\bigl(\delta^2_\bd\bigr)/\im\bigl(\delta^1_\bd\bigr)}
            \\
            & & \EHb^2(G;\R) \ar@{}[r]|-{\gray=} \ar@{ `->}[u]
                & \graylap{\ker(c)}
            %
            & \hspace*{1.5cm} % alignment hack :/
        }
    \]
    (wobei $c$ die Vergleichsabbildung ist, $\pi$ und $\pi_\bd$ die kanonischen
    Projektionen sind und die linke Inklusion im Sinne von \cref{setup} zu
    verstehen ist). Durch eine genaue Untersuchung des Kerns und Bilds
    von~$\phi$ erhält man die Behauptung aus dem Isomorphiesatz.
    \\
\end{proofsketch}
