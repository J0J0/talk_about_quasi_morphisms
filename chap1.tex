\chapter{Quasimorphismen und der Zusammenhang zu beschränkter Kohomologie}
\begin{thSetup}
    Sei $G$ in diesem Kapitel stets eine Gruppe.
\end{thSetup}

\begin{thDef}[Quasimorphismus, Defekt] \hfill
    \begin{itemize}
        \item
            Ein \emph{Quasimorphismus (auf $G$)} ist eine
            Abbildung $f\colon G\to\R$ mit
            \[ \sup_{g,h\in G} \abs{f(g) + f(h) - f(gh)} \;<\, \infty  . \]

        \item
            Der \emph{Defekt~$D(F)$} eines Quasimorphismus $f\colon G\to\R$
            ist
            \[ D(f) \defeq \inf\bigl\{ c\in\R[\geq0] \Mid
              \sup\nolimits_{g,h\in G}\, \abs{f(g) + f(h) - f(gh)} \leq c \bigr\}
            . \]
    \end{itemize}
\end{thDef}

\begin{BspList}
\item
    Ein Quasimorphismus $G\to\R$ hat genau dann Defekt~$0$, wenn er ein
    Gruppenhomomorphismus $G\to\R$ ist.

\item
    Jede beschränkte Abbildung $f\colon G\to\R$ ist ein
    Quasimorphismus mit $D(f) \leq 3\norm{f}_\supr$.

\item
    Allgemeiner ist für jeden Gruppenhomomorphismus $f\colon G\to\R$ und
    jede beschränkte Abbildung $b\colon G\to\R$ die (punktweise) Summe $f+b$
    ein Quasimorphismus.
\end{BspList}

Das letzte Beispiel gibt Anlass zu folgender Definition:

\begin{thDef}[Trivialer Quasimorphismus]
    Sei $f\colon G\to\R$ ein Quasimorphismus. Dann ist $f$ ein \emph{trivialer
    Quasimorphismus}, falls es einen Gruppenhomomorphismus $f'\colon G\to\R$
    mit $\norm{f-f'}_\supr < \infty$ gibt.
\end{thDef}

Offensichtlich bildet die Menge aller Quasimorphismen (bzw. aller trivialen
Quasimorphismen) einen $\R$-Vektorraum.

\begin{thDef}[Vektorraum aller (trivialen) Quasimorphismen]
    Wir schreiben $\QM(G)$ (bzw. $\QM_0(G)$) für den $\R$-Vektorraum aller
    Quasimorphismen (bzw. aller trivialen Quasimorphismen). 
\end{thDef}

\begin{thProposition}[Zerlegung von $\QM_0$]
    Es gilt:
    \[ \QM_0(G) = \Hom_\Z(G,\R) \oplus \Cb^1(G)  . \]
\end{thProposition}

\begin{thBeispiel}[Wörterzählende Quasimorphismen]
    Nicht-triviale Beispiele liefern sogenannte \enquote{counting
    quasimorphisms}, welche wir im Folgenden auf Deutsch als
    \emph{wörterzählende Quasimorphismen} bezeichnen. Sei $F$ die freie Gruppe
    über einer Menge~$S$. Elemente aus $F$ seien stets reduzierte Wörter in den
    Erzeugern aus~$S$. Ist nun $w\in F$, so kann man zeigen, dass die Abbildung
    \begin{align*}
        \psi_w\colon F &\to \R  \\
        g &\mapsto (\text{Anzahl der Vorkommen von $w$ in $g$})
                 - (\text{Anzahl der Vorkommen von $w^{-1}$ in $g$})
    \end{align*}
    einen Quasimorphismus auf~$F$ liefert. Einen Beweis dieser Behauptung findet
    man beispielsweise bei
    Löh\cite[Ch.\,2,\;2.5.3,\;Lemma~2.5.11]{lecnotes:loeh:bdcoho}.
    Für eine ausführlichere Behandlung (auch anderer Varianten) wörterzählender
    Quasimorphismen siehe Caligari\cite[Ch.\,2,\;2.3.2]{bookc:calegari09}.
\end{thBeispiel}

\begin{thErinnerDef}[Vergleichsabbildung] \hfill
    \begin{itemize}
        \item
            Die Inklusion von Kokettenkomplexen $\Cb^\ast(G;\R) \hookrightarrow
            C^\ast(G;\R)$ induziert eine Abbildung $\Hb^\ast(G;\R) \to
            H^\ast(G;\R)$ in Kohomologie.
        \item
            Es ist
            \[ \EHb^\ast(G;\R)
                \defeq \ker\bigl( \Hb^\ast(G;\R) \to H^\ast(G;\R) \bigr)
            \]
            der (gradweise) Kern der Vergleichsabbildung.
    \end{itemize}
\end{thErinnerDef}

\begin{thSatz}[Zusammenhang zwischen Quasimorphismen und Kohomologie]
    Es gilt:
    \[ \Quot{\QM(G)}{\QM_0(G)} \;\cong\; \EHb^2(G;\R)  . \]
\end{thSatz}

\begin{thKorollar}
    Verschwindet die zweite Kohomologie $H^2(G;\R)$ von $G$, so gilt
    \[ \Quot{\QM(G)}{\QM_0(G)} \;\cong\; \Hb^2(G;\R)  . \]
\end{thKorollar}

\begin{thBeispiel}[Quasimorphismen auf freien Gruppen]
    Sei $F$ eine freie Gruppe vom Rang mindestens~$2$. Aus dem vorherigen
    Vortrag wissen wir, dass $\dim_{\R} \Hb^2(G;\R) = \infty$ gilt. Außerdem
    kann man zeigen, dass $H^2(G;\R)$ verschwindet (siehe zum Beispiel
    Löh\cite[Ch.\,1,\;1.3.4,\;Example~1.3.13]{lecnotes:loeh:bdcoho}).
    Dann folgt aus dem Korollar sofort
    \[ \dim_{\R} \Quot{\QM(G)}{\QM_0(G)} = \infty  , \]
    d.\,h. es gibt unendlich viele (linear unabhängige) nicht-triviale
    Quasimorphismen auf~$F$.
\end{thBeispiel}
