
\section{Quasimorphismen auf freien Produkten %
    \unexpanded{\small(nach Rolli\cite{arxiv:rolli09})}}
\begin{thKonstruktion}
    \label{freep:construction}
    %
    Sei $G \defeq \freeprod_{i\in I} G_i$ das freie Produkt
    einer mindestens zweielementigen Familie nicht-trivialer Gruppen
    und sei $()\in G$ das leere Wort.
    %
    Sei $\hatCb^1(G_i;\R) \subset \Cb^1(G_i;\R)$ der Unterraum aller ungeraden
    beschränkten Quasimorphismen und sei
    \[ \textstyle
        V_0(G) \defeq \left\{ (\sigma_i)_{i\in I} \in \prod_{i\in I}
        \smash{\hatCb^1(G_i;\R)} \Mid
        \sup_{i\in I}\, \norm{\sigma_i}_\supr < \infty \right\}
    . \]%
    \rule{0pt}{1.3\ht\strutbox}%
    Für $\sigma = (\sigma_i)_{i\in I} \in V_0(G)$ sei
    $g_\sigma\colon G \to \R$ gegeben durch
    \[
        x \mapsto 
        \begin{cases}
            0, \quad& \begin{aligned} &\text{falls $x=()$},\end{aligned}
                      % ^ hack for equal spacing
            \\
            \ksum^n \sigma_{i_k}(x_k), \quad&
            \smash{%
            \begin{aligned}[t]
                &\text{falls $x = x_1\cdots x_n$ die eindeutige reduzierte} \\[-5pt]
                &\text{Darstellung von $x$ mit $x_k\in G_{i_k}$ für alle $k$ ist.}
            \end{aligned}%
            }
        \end{cases}
    \]
\end{thKonstruktion}
\medskip

\begin{thProposition}[Quasimorphismen auf freien Produkten]
    \label{freep:qmonfreep}%
    %
    Die Abbildung $g_\sigma$ aus \cref{freep:construction} ist für
    alle $\sigma\in V_0(G)$ ein Quasimorphismus.
\end{thProposition}

\begin{proofsketch}
    Dies folgt mit einer genauen Betrachtung, was bei der Verknüpfung zweier
    Wörter aus~$G$ passieren kann, und der Konstruktion von~$V_0(G)$ aus einer
    einfachen Rechnung.
    \\
\end{proofsketch}

\begin{thProposition}
    \label{freep:qmtocoho}%
    %
    Die Abbildung
    \[ V_0(G) \to \EHb^2(G;\R), \quad
        \sigma \mapsto \bigl[\delta^1(g_\sigma)\bigr]_\bd
    \]
    (mit $g_\sigma$ aus \cref{freep:construction})
    ist wohldefiniert, linear und injektiv.
\end{thProposition}

\begin{thKorollar}[Beschränkte Kohomologie freier Gruppen ist nicht trivial]
    \label{freep:cohofreegrp}%
    %
    Sei $F$ eine freie Gruppe vom Rang mindestens~$2$. Dann gilt
    $\dim_{\R} \Hb^2(F;\R) = \infty$.
\end{thKorollar}
