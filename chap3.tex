\chapter{Quasimorphismen auf freien Produkten}
\begin{thKonstruktion}
    \label{freep:construction}
    %
    Sei $I$ eine Menge mit $\abs{I}\geq 2$ und $(G_i)_{i\in I}$ eine Familie
    nicht-trivialer Gruppen. Sei weiter
    \[ G \defeq \freeprod_{i\in I} G_i \]
    das freie Produkt der Gruppen. Wir behandeln jede der Gruppen~$G_i$
    unter der üblichen Identifikation als Untergruppe von~$G$.
    Ist nun $x\in G$, so gibt es ein $n\in\N$ und Indizes $i_1,\dots,i_n\in I$, so
    dass $x$ eine eindeutige Darstellung
    \[ x = x_1 \cdots x_n \]
    besitzt, wobei $x_j \in G_{i_j}$ nicht-trivial für alle $j\in\setOneto n$
    und $i_j\neq i_{j+1}$ für alle $j\in\setOneto{n-1}$.
    
    Wir möchten nun Quasimorphismen auf~$G$ konstruieren. Für $i\in I$ sei dazu
    $\hatCb^1(G_i)$ derjenige Unterraum von $\Cb^1(G_i)$, der alle ungeraden
    beschränkten Quasimorphismen enthält. Weiter sei
    \[ V(G) \defeq \prod_{i\in I} \hatCb^1(G_i) \]
    und
    \[ V_0(G) \defeq \bigl\{ (\sigma_i)_{i\in I} \in V(G) \Mid
        \sup\nolimits_{i\in I}\, \norm{\sigma_i}_\supr < \infty \bigr\}
    . \]%
    \rule{0pt}{1.3\ht\strutbox}%
    Für $\sigma = (\sigma_i)_{i\in I} \in V_0(G)$ definieren wir nun eine
    Abbildung~$g_\sigma$ wie folgt:
    \begin{align*}
        g_\sigma\colon G &\to \R
        \\
        x &\mapsto \isum^n \sigma_{i_j}(x_j), \quad
        \text{ falls $x = x_1\cdots x_n$ im obigen Sinne}
    . \end{align*}
\end{thKonstruktion}

\begin{thProposition}[Quasimorphismen auf freien Produkten]
    Die Abbildung $g_\sigma$ aus \cref{freep:construction} ist für alle
    $\sigma\in V_0(G)$ ein Quasimorphismus.
\end{thProposition}

\begin{thProposition}
    Die Abbildung
    \[ V_0(G) \to \EHb^2(G;\R), \quad
        \sigma \mapsto [\partial g_\sigma]_\bd
    \]
    ist linear und injektiv.
\end{thProposition}

\begin{thKorollar}[Beschränkte Kohomologie freier Gruppen]
    Sei $F$ eine freie Gruppe vom Rang mindestens~$2$. Dann gilt
    \[ \dim_{\R} \Hb^2(G;\R) = \infty  . \]
\end{thKorollar}

\begin{thKorollar}[Beschränkte Kohomologie von $\operatorname{PSL}_2(\Z)$]
    Die beschränkte Kohomologie $\Hb^2( \operatorname{PSL}_2(\Z); \R )$ von
    $\operatorname{PSL}_2(\Z)$ ist nicht trivial.
\end{thKorollar}
