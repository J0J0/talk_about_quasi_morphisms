\chapter{Quasimorphismen auf freien Produkten} \label{freep}
In diesem Kapitel wollen wir eine Konstruktion angeben, die (eine potentiell
große Anzahl, vgl. \cref{freep:cohofreegrp}) von Quasimorphismen auf freien
Produkten und insbesondere freien Gruppen liefert. Der Großteil dieses Kapitels
basiert auf Rolli\cite{arxiv:rolli09}.

\begin{thKonstruktionSetup}
    \label{freep:construction}
    %
    Sei $I$ eine Menge mit $\abs{I}\geq 2$ und $(G_i)_{i\in I}$ eine Familie
    nicht-trivialer Gruppen. Sei weiter
    \[ G \defeq \freeprod_{i\in I} G_i \]
    das freie Produkt der Gruppen. Wir behandeln jede der Gruppen~$G_i$
    unter der üblichen Identifikation als Untergruppe von~$G$ und bezeichnen
    mit $()\in G$ das leere Wort.
    Ist nun $x\in G\setminus\{()\}$, so gibt es ein $n\in\N$ und Indizes
    $i_1,\dots,i_n\in I$, so dass $x$ eine eindeutige Darstellung
    \[ x = x_1 \cdots x_n \]
    besitzt, wobei folgende Bedingungen gelten: Für alle $k\in\setOneto n$ ist
    $x_k \in G_{i_k}$ nicht-trivial und für alle $k\in\setOneto{n-1}$ gilt
    $i_k\neq i_{k+1}$.
    
    Wir möchten nun Quasimorphismen auf~$G$ konstruieren. Für $i\in I$ sei dazu
    $\hatCb^1(G_i;\R)$ derjenige Unterraum von $\Cb^1(G_i;\R)$, der alle ungeraden
    beschränkten Quasimorphismen enthält. Weiter sei
    \[ V(G) \defeq \prod_{i\in I} \hatCb^1(G_i;\R) \]
    und
    \[ V_0(G) \defeq \bigl\{ (\sigma_i)_{i\in I} \in V(G) \Mid
        \sup\nolimits_{i\in I}\, \norm{\sigma_i}_\supr < \infty \bigr\}
    . \]%
    \rule{0pt}{1.3\ht\strutbox}%
    Für $\sigma = (\sigma_i)_{i\in I} \in V_0(G)$ definieren wir nun eine
    Abbildung~$g_\sigma$ wie folgt:
    \begin{align*}
        g_\sigma\colon G &\to \R
        \\
        x &\mapsto 
        \begin{cases}
            0, \quad& \text{falls $x=()$},
            \\
            \ksum^n \sigma_{i_k}(x_k), \quad&
            \text{falls $x = x_1\cdots x_n$ im obigen Sinne} .
        \end{cases}
    \end{align*}
\end{thKonstruktionSetup}

\begin{thProposition}[Quasimorphismen auf freien Produkten]
    \label{freep:qmonfreep}%
    %
    Die Abbildung $g_\sigma$ aus \cref{freep:construction} ist für
    alle $\sigma\in V_0(G)$ ein Quasimorphismus.
\end{thProposition}

\begin{proof}
    Seien $x,y\in G$ mit Darstellungen $x = x_1\cdots x_n$ bzw.
    $y = y_1\cdots y_m$ und zugehörigen Indizes $i_1,\ldots,i_n$ bzw.
    $j_1,\ldots,j_m$ wie in \cref{freep:construction}. Gilt nun $i_n \neq j_1$,
    so hat $xy\in G$ einfach die Darstellung $x_1\cdots x_n y_1\cdots y_m$.
    Gilt andererseits $i_n = j_1$, so ist $z \defeq x_ny_1$ entweder
    nicht-trivial in $G_{i_n} = G_{j_1}$ und damit hat $xy$ die Darstellung
    \[ x_1\cdots x_{n-1} \, z \, y_2\cdots y_m  , \]
    oder aber $z$ ist das neutrale Element in $G_{i_n}$, womit $x_n$ gegen
    $y_1$ wegfällt und wir $x_{n-1} y_2$ nach demselben Prinzip betrachten
    müssen. Da beide Darstellungen endlich sind, muss dieser Prozess aber
    irgendwann aufhören und wir erhalten für $xy$ eine Darstellung
    \[ x_1\cdots x_{n-r} \, z \, y_{r+1}\cdots y_m  , \]
    mit $r\in\N$ und $z\in G_{i_{n-r+1}} = G_{j_r}$ oder aber $z=()$.
    In jedem Fall gilt aber $x_{n-k} = y_{k+1}^{-1}$ für alle
    $k\in\setZeroto{r-2}$, und weil $\sigma$ nach Konstruktion aus ungeraden
    Quasimorphismen besteht, gilt
    \[ \sigma_{i_{n-k}}(x_{n-k}) + \sigma_{j_{k+1}}(y_{k+1}^{-1}) = 0 \]
    für alle $k\in\setZeroto{r-2}$. Damit rechnen wir:
    \begin{align*}
        \abs[\big]{ g_\sigma(x) + g_\sigma(y) - g_\sigma(xy) }
        &= \abs[\Big]{
                \ksum^n \sigma_{i_k}(x_k) + \ksum^m \sigma_{j_k}(y_k)
                - \ksum^{n-r} \sigma_{i_k}(x_k)
                - \sigma_{j_r}(z)
                - \ksum[r+1]^m \sigma_{j_k}(y_k)
            }
        \\[0.5ex]
        &= \abs[\big]{
            \sigma_{i_{n-r+1}}(x_{n-r+1})
            + \sigma_{j_r}(y_r)
            - \sigma_{j_r}(z)
            }
        \\[0.5ex]
        &\leq \norm{\sigma_{i_{n-r+1}}}_\supr
            + \norm{\sigma_{j_r}}_\supr
            + \norm{\sigma_{j_r}}_\supr
        \\[0.5ex]
        &\leq 3\cdot\sup_{i\in I}\, \norm{\sigma_i}_\supr
    \end{align*}
    Es folgt
    \[ D(g_\sigma) \leq 3\cdot\sup_{i\in I}\, \norm{\sigma_i}_\supr < \infty \]
    und damit ist $g_\sigma$ wie behauptet ein Quasimorphismus auf~$G$.
    \\
\end{proof}

Offenbar ist $V_0(G)$ ein Untervektorraum von $V(G)$, womit folgende Behauptung
Sinn ergibt:

\begin{thProposition}
    \label{freep:qmtocoho}%
    %
    Die Abbildung
    \[ V_0(G) \to \EHb^2(G;\R), \quad
        \sigma \mapsto \bigl[\delta^1(g_\sigma)\bigr]_\bd
    \]
    (mit $g_\sigma$ aus \cref{freep:construction})
    ist wohldefiniert, linear und injektiv.
\end{thProposition}

\begin{proof}
    Wie im Beweis von \cref{qmor:qmcoho} sieht man, dass die Abbildung
    wohldefiniert ist und die Linearität ist klar nach Konstruktion.
    Sei $\sigma\in V_0(G)$ mit
    $[\delta^1(g_\sigma)]_\bd = 0 \in\Hb^2(G;\R)$. Daraus folgt aber
    (wie im Beweis von \cref{qmor:qmcoho}), dass $g_\sigma - b = f$
    für ein $b\in\Cb^1(G;\R)$ und ein $f\in\Hom_\Z(G,\R)$ gilt. Seien
    $i\in I$, $x\in G_i$ und $k\in\N$. Dann gilt
    \[ g_\sigma(x^k) - b(x^k) = \sigma_i(x^k) - b(x^k) 
        \leq \norm{\sigma_i}_\supr + \norm{b}_\supr
    , \]
    womit
    \[ f(x) \leq \frac{\norm{\sigma_i}_\supr + \norm{b}_\supr}{k}
        \;\xrightarrow[k\to\infty]{}\; 0
    , \]
    also $f(x) = 0$ folgt. Weil $\bigcup_{i\in I} G_i \subset G$ ein
    Erzeugendensystem von $G$ ist, gilt somit schon $f=0$, d.\,h.
    $g_\sigma = b \in\Cb^1(G;\R)$. Seien nun $i,j\in I$ mit $i\neq j$ und
    seien $x\in G_i,\; y\in G_j$. Für alle $k\in\Z$ gilt dann
    \[ g_\sigma\bigl( (xy^{\pm1})^k \bigr)
        = k\cdot \bigl( \sigma_i(x) \pm \sigma_j(y) \bigr)
    . \]
    Weil $g_\sigma = b$ beschränkt ist, folgt 
    \[ \sigma_i(x) \pm \sigma_j(y) = 0
        \qtextq{und daraus}
        \sigma_i(x) = 0 = \sigma_j(y)
    . \]
    Da $i,j,x,y$ beliebig waren, folgt $\sigma = 0$. Damit ergibt sich
    die Injektivität der Abbildung aus der Behauptung.
    \\
\end{proof}

\begin{thKorollar}[Beschränkte Kohomologie freier Gruppen ist nicht trivial]
    \label{freep:cohofreegrp}%
    %
    Sei $F$ eine freie Gruppe vom Rang mindestens~$2$. Dann gilt
    \[ \dim_{\R} \Hb^2(F;\R) = \infty  . \]
\end{thKorollar}

\begin{proof}
    Sei $S\subset F$ ein freies Erzeugendensystem von~$F$. Dann gilt
    $F \cong \freeprod_S \Z$, womit es nach \cref{freep:qmtocoho} genügt,
    zu zeigen, dass der Raum $\hatCb^1(\Z;\R)$ unendlich-dimensional ist. Man
    sieht aber leicht, dass
    \begin{align*}
        \ell^\infty(\R) &\to \hatCb^1(\Z;\R)
        \\
        (x_n)_{n\in\N} &\mapsto \left( 
        \begin{aligned}
            \Z &\to \R \\
             k &\mapsto \sgn(k) \, x_{\abs{k}}
        \end{aligned}
        \,\right)
    \end{align*}
    ein Isomorphismus zwischen dem unendlich-dimensionalen $\R$-Vektorraum der
    reellwertigen beschränkten Folgen und $\hatCb^1(\Z;\R)$ ist. (Dabei ist
    \enquote{$\sgn$} die Signumsfunktion.)
    \\
\end{proof}

\begin{thBeispiel}%
    [Beschränkte Kohomologie von $\operatorname{PSL}_2(\Z)$ im Grad~2]
    %
    Wir betrachten die Gruppe $\operatorname{PSL}_2(\Z) \cong \txtZQuot2 \ast
    \txtZQuot3$. Um zu zeigen, dass $\Hb^2( \operatorname{PSL}_2(\Z); \R )$
    nicht trivial ist, genügt es nach \cref{freep:qmtocoho} zu zeigen, dass 
    \[ V_0\bigl(\operatorname{PSL}_2(\Z)\bigr)
        \cong \hatCb^1(\txtZQuot2;\R) \times \hatCb^1(\txtZQuot3;\R)
    \]
    nicht trivial ist. Sei $n\in\N$. Ein ungerader (notwendigerweise
    beschränkter) Quasimorphismus auf $\txtZQuot n$ ist offensichtlich
    durch die Werte auf den Restklassen von $1$ bis $\lfloor (n-1)/2 \rfloor$
    eindeutig bestimmt. Für $\txtZQuot2$ gilt also $\hatCb^1(\txtZQuot2;\R) =
    \{0\}$. Aber für $\txtZQuot3$ können wir den Wert auf der Restklasse von~$1$
    frei vorgeben, so dass wir 
    \[ \hatCb^1(\txtZQuot3;\R) \cong \R \qtextq{und damit}
        V_0\bigl(\operatorname{PSL}_2(\Z)\bigr) \cong \R
    \]
    erhalten.
\end{thBeispiel}
