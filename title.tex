%
\begin{tikzpicture}[remember picture,overlay]
    \node [yshift=-1.3cm,color=black!40] at (current page.north)
        {%
        \hfill%
        Seminar \enquote{Beschränkte Kohomologie} im
        SS~2014 an der Universität Regensburg%
        \hfill\mbox{}%
        };
\end{tikzpicture}

\vspace*{-0.5cm}
\begin{center}
    \Large Quasimorphismen
\end{center}

\medskip\noindent
J. Prem (\href{mailto:Johannes.Prem@stud.uni-regensburg.de}%
{\texttt{Johannes.Prem@stud.uni-regensburg.de}})
\hfill
06.~Mai 2014
\\[-8pt]
\rule{\textwidth}{0.4pt}

\smallskip\noindent
%
Ein \emph{Quasimorphismus} ist eine Abbildung, die
\enquote{bis auf einen beschränkten Fehler ein Gruppenhomomorphismen ist}.
In diesem Vortrag werden wir sehen, wie Quasimorphismen helfen können,
beschränkte Kohomologie von Gruppen zu berechnen.
