\documentclass[11pt,a4paper,ngerman,DIV=11,bibliography=totoc,titlepage=false]{scrreprt}

%%%%%%%%%%%%%%%%%%%%%%%%%%%%%%%%%%%%%%%%%%%%%%%%%%%%%%%%%%%%%%%%%%%%%
%%% packages
%%%%%%%%%%%%%%%%%%%%%%%%%%%%%%%%%%%%%%%%%%%%%%%%%%%%%%%%%%%%%%%%%%%%%

\usepackage[utf8]{inputenc}
\usepackage[T1]{fontenc}
\usepackage[ngerman]{babel}

\usepackage{amsmath}
\usepackage{amssymb}
\usepackage{amsthm}
\usepackage{mathtools}
\usepackage[all]{xy}
\usepackage{tikz}

\usepackage[babel]{csquotes}
\usepackage[shortlabels]{enumitem}
\usepackage[numbers,sort&compress]{natbib}
\usepackage{ifmtarg}
\usepackage{xstring}
\usepackage{remreset}


\usepackage[pdftex,bookmarks,colorlinks=false,pdfborder={0 0 0},%
            pdftitle={Seminar Beschränkte Kohomologie - %
                      Vortrag 4: Quasimorphismen},%
            pdfauthor={Johannes Prem}]{hyperref}
%
\usepackage{cleveref}
\let\cref=\Cref

\usepackage{myhelpers}  % my own myhelpers.sty
\usepackage{mymathmisc} % my own mymathmisc.sty


%%%%%%%%%%%%%%%%%%%%%%%%%%%%%%%%%%%%%%%%%%%%%%%%%%%%%%%%%%%%%%%%%%%%%
%%% macro definitions and other things
%%%%%%%%%%%%%%%%%%%%%%%%%%%%%%%%%%%%%%%%%%%%%%%%%%%%%%%%%%%%%%%%%%%%%

% don't reset footnote numbers
\makeatletter
\@removefromreset{footnote}{chapter}
\makeatother


% make parenthesized versions of \ref and cleveref's \cref
\newcommand*{\pref}[1]{(\ref{#1})}
\newcommand*{\pcref}[1]{(\cref{#1})}

% make a even more clever \mycref that produces "Lemma 42a)" etc.
% (to see it in action check out the code in chap2.tex) 
\newcommand{\mycref}[1]{%
    \begingroup%
    \StrCount{#1}{:}[\mycrefCount]%
    \StrBefore[\mycrefCount]{#1}{:}[\myrefMain]%
    \expandafter\cref\expandafter{\myrefMain}\,\ref{#1}%
    \endgroup%
}

% make \varepsilon and \varphi default
\varifygreekletters{\epsilon\phi}

% change the qedsymbol to my favoured blacksquare
\renewcommand{\qedsymbol}{$\blacksquare$}

% style for /all/ theorem like environments
\newtheoremstyle{mythms}
 {15pt}% space above
 {12pt}% space below 
 {}% body font
 {}% indent amount
 {\bfseries}% theorem head font
 {.}% punctuation after theorem head
 {0.6cm plus 0.25cm minus 0.1cm}% space after theorem head (\newline possible)
 {}% theorem head spec 
 
% set style and define thm like environments
\theoremstyle{mythms}
\newtheorem{globalnum}{DUMMY DUMMY DUMMY}[chapter]
\newtheorem{thDef}[globalnum]{Definition}
\newtheorem{thNotation}[globalnum]{Notation}
\newtheorem{thSatz}[globalnum]{Satz}
\newtheorem{thProposition}[globalnum]{Proposition}
\newtheorem{thLemma}[globalnum]{Lemma}
\newtheorem{thKorollar}[globalnum]{Korollar}

\newtheorem{thSetup}[globalnum]{Setup}
\newtheorem{thErinnerung}[globalnum]{Erinnerung}
\newtheorem{thErinnerDef}[globalnum]{Erinnerung/Definition}
\newtheorem{thBemerkung}[globalnum]{Bemerkung}
%\newtheorem{thWarnung}[globalnum]{Warnung}
\newtheorem{thBeispiel}[globalnum]{Beispiel}
\newtheorem{thBeispiele}[globalnum]{Beispiele}
\newenvironment{BspList}[1][]{%
\nopagebreak\begin{thBeispiele}#1%
\hfill\begin{enumerate}[(a),parsep=0pt,itemsep=0.8ex,leftmargin=2em]%
}{%
\end{enumerate}\end{thBeispiele}
}
%

% also define a 'proofsketch' version of 'proof'
\newenvironment{proofsketch}[1][]{%
\begin{proof}[Beweisskizze#1]
}{%
\end{proof}
}

% inject pdfbookmarks at thm like environments
\makeatletter
\let\origthmhead=\thmhead
\renewcommand{\thmhead}[3]{%
\origthmhead{#1}{#2}{#3}%
\belowpdfbookmark{#1\@ifnotempty{#1}{ }#2\thmnote{ (#3)}}{#1#2}%
}
\makeatother

% new math operators
\DeclareMathOperator*{\bigdotcup}{\overset{\mkern0mu\scalebox{0.6}{$\bullet$}}{\bigcup}}

% avoid "already defined"
\let\ker=\relax
% new math 'operators'
\newcommand{\sDMO}[1]{\expandafter\DeclareMathOperator\csname#1\endcsname{#1}}

\sDMO{const}
\sDMO{Hom}
\sDMO{id}
\sDMO{im}
\sDMO{ker}
\sDMO{QM}

% make quantors that use \limits per default
\DeclareMathOperator*{\Exists}{\exists}
\DeclareMathOperator*{\forAll}{\forall}

% define an 'abs', 'norm' and 'Spann' command
\DeclarePairedDelimiter{\abs}{\lvert}{\rvert}
\DeclarePairedDelimiter{\norm}{\lVert}{\rVert}
\DeclarePairedDelimiter{\Spann}{\langle}{\rangle}

% define missing arrows
\newcommand{\longto}{\longrightarrow}
\newcommand{\longhookrightarrow}{\lhook\joinrel\relbar\joinrel\rightarrow}
\newcommand{\isorightarrow}[1][]{\xrightarrow[#1]{\smash{\raisebox{-2pt}{$\sim$}}}}

% provide mathbb symbols \N \Z \Q \R and \C
\defmathbbsymbols{Z Q C}
\defmathbbsymbolsubs{N R}

% define some set specific macros
\newcommand{\setclosure}[1]{\overline{#1}}
\newcommand{\setinterior}[1]{#1^\circ}
\newcommand{\setboundary}[1]{\partial #1}

% quotient by means of groups/rings/vector spaces
\newcommand{\Quot}[3][\big]{%
\raisebox{2pt}{$\mathsurround=0pt\displaystyle #2$}\mkern-3mu%
#1/%
\mkern-3mu\raisebox{-3.5pt}{$\displaystyle #3$}%
}
\newcommand{\QuotS}[3][]{%
\raisebox{2pt}{$\mathsurround=0pt\displaystyle #2$}\mkern-1mu%
#1/%
\mkern-3mu\raisebox{-3.5pt}{$\displaystyle #3$}%
}

% just some shortcuts
\newcommand{\Cb}{C_{\mathsf{b}}}
\newcommand{\defeq}{\coloneqq}
\newcommand{\eqdef}{\eqqcolon}
\newcommand{\Hb}{H_{\mathsf{b}}}
\newcommand{\half}{\frac{1}{2}}
\newcommand{\homo}{^\mathsf{h}}
\newcommand{\isum}[1][1]{\sum_{i=#1}}
\newcommand{\mr}{\mathrm}
%\newcommand{\mt}{^\mathsf{t}}
%\newcommand{\Nfolge}[1]{\left(#1_n\right)_{n\in\N}}
\newcommand{\setOneto}[1]{\{1,\ldots,#1\}}
\newcommand{\setZeroto}[1]{\{0,\ldots,#1\}}
\newcommand{\supr}{\infty}
\newcommand{\thalf}{\tfrac{1}{2}}

% some text shortcuts
\qXq{iff}
\qXq{implies}
\qTXq{oder}
\qTXq{und}
\qqTXqq{und}
\newcommand{\fuer}[1][\qquad]{#1\text{für }}

%
\newcommand{\Achtung}{\emph{Achtung:} }

%% use some tikz libraries
%\usetikzlibrary{arrows,calc,intersections}

% xy tip selection (ComputerModern)
\SelectTips{cm}{}
\UseTips

% listing with -- is nicer than with bullets 
\setlist[itemize,1]{label=--}

% start at chapter 0
\setcounter{chapter}{-1}

%%%%%%%%%%%%%%%%%%%%%%%%%%%%%%%%%%%%%%%%%%%%%%%%%%%%%%%%%%%%%%%%%%%%%
%%% document
%%%%%%%%%%%%%%%%%%%%%%%%%%%%%%%%%%%%%%%%%%%%%%%%%%%%%%%%%%%%%%%%%%%%%

\begin{document}


\subject{Seminar: Beschränke Kohomologie}
\title{Quasimorphismen}
\author{Johannes Prem}
\date{06.05.2014}

\maketitle
\thispagestyle{empty}

\vfill
\begin{center}\footnotesize%
    Version vom \today.
    
    \medskip
    (Der \LaTeX-Quellcode für dieses Skript und das zugehörige Handout befindet
    sich auf github: 
    \url{https://github.com/J0J0/talk_about_quasi_morphisms}\,)
\end{center}
\newpage


\chapter{Vorwort, Notation und Vorbereitungen}
\section{Einleitung}
\[ \dots \] % TODO

\section{Notation}
In diesem Skript wird folgende Notation verwendet:
\begin{itemize}
    \item
        Sowohl $\subset$ als auch $\subseteq$ stehen für: enthalten oder gleich.
        Echt enthalten wird durch $\subsetneq$ gekennzeichnet.
    
    \item
        Die \emph{natürlichen Zahlen $\N$} beginnen mit $0$.

    \item
        Ist $f\colon X\to\R$ eine Abbildung, so bezeichnet
        $\norm{f}_\supr = \sup_{x\in X}\,\abs{f(x)}$ die
        Supremumsnorm von~$f$.

    \item
        Für einen Ring~$R$ seien $\lCh R$ bzw. $\lCoCh R$
        die Kategorien der Kettenkomplexe bzw. Kokettenkomplexe, jeweils
        $\Z$-indiziert und in $\lMod R$, der Kategorie der Links-$R$-Moduln.
        Außerdem bezeichnet $\lGrad R$ die Kategorie der $\Z$-graduierten
        Links-$R$-Moduln.

    \item
        Sind $V$ und $W$ zwei normierte $\R$-Vektorräume, so ist
        $B(V,W)$ der bezüglich der Opertornorm normierte $\R$-Vektorraum
        aller stetig linearen Operatoren $V\to W$.
\end{itemize}

\section{Grundlegende Definitionen und Vorüberlegungen}
\label{ch3:basics}%
%
Sei $G$ eine Gruppe. Wir bezeichnen den \emph{Bar-Kettenkomplex} von~$G$
mit $C_\ast(G) = (C_\ast,\partial_\ast)$, wobei
\[ C_k(G) = \bigoplus_{g\in G^k} [g_1|\ldots|g_k]\,\Z \]
für alle $k\in\N$ (und -- falls wir den Komplex $\Z$-indiziert betrachten wollen
-- $C_k(G) = 0$ für $k\in\Z_{<0}$) sowie
\begin{align*}
    \partial_k\colon C_k(G) &\to C_{k-1}(G)
    \\
    [g_1|\ldots|g_k] &\mapsto [g_2|\ldots|g_k]
    + \isum^{k-1} (-1)^i \, [g_1|\ldots|g_ig_{i+1}|\ldots|g_k]
    + (-1)^k \, [g_1|\ldots|g_{k-1}]
\end{align*}
(auf Erzeugern) für alle $k\in\N$. Der \emph{Bar-Kokettenkomplex mit reellen
Koeffizienten} ist der (algebraisch) dualisierte Kokettenkomplex
\[ C^\ast(G;\R) = \Hom_\Z\bigl( C_\ast(G), \R \bigr)  \;\in\lCoCh{\R} , \]
wobei wir die Folge der induzierten Korandoperatoren mit~$\delta$ bezeichnen.
Der \emph{reelle Bar-Kettenkomplex}
\[ C_\ast^{\R}(G)
    \defeq C_\ast(G) \underset{\scriptscriptstyle\Z}\otimes \R
    \;\in\lCh{\R}
\]
wird zusammen mit der $\ell^1$-Norm bezüglich der Basis zu einem
normierten Kettenkomplex. Der topologisch dualisierte Kokettenkomplex
\[ \Cb^\ast(G;\R) \defeq B\bigl( C_\ast^{\R}(G), \R \bigr) \]
ist der \emph{Banach-Bar-Kokettenkomplex} zu~$G$ und
besteht aus Banachräumen bezüglich der Operatornorm.

Die \emph{(gewöhnliche) (Gruppen-)Kohomologie} von $G$ ist
\[ H^\ast(G;\R) = H^\ast\bigl( C^\ast(G;\R) \bigr)  , \]
wobei $H^\ast\colon\lCh{\R}\to\lGrad{\R}$ den algebraischen Kohomologiefunktor
bezeichnet. Die \emph{beschränkte (Gruppen-)Kohomologie} von $G$ ist
\[ \Hb^\ast(G;\R) = H^\ast\bigl( \Cb^\ast(G;\R) \bigr)  . \]

Nach Definition gilt $C_1(G) = \bigoplus_{g_1\in G}\, [g_1]\,\Z$. Fassen wir
$G$ vermöge der Inklusion auf die entsprechenden Basiselemente aus $C_1(G)$
als Teilmenge von $C_1(G)$ auf, so ist ein Kokettenelement $f\in C^1(G;\R)$ also 
eindeutig durch die Werte auf $G$ bestimmt. Dies liefert einen
$\R$-Vektorraum-Isomorphismus 
\[ C^1(G;\R) \cong \R^G  \]
zwischen $C^1(G;\R)$ und dem Raum aller $\R$-wertigen Funktionen auf~$G$.
Wir werden daher (etwas schlampig) im Folgenden nicht zwischen den beiden
$\R$-Vektorräumen unterscheiden und $C^1(G;\R)$ jeweils dem Kontext entsprechend
interpretieren.

Ähnlich gehen wir bei $\Cb^1(G;\R)$ vor: Ein Element $f\in\Cb^1(G;\R)$ ist ein
stetig lineares Funktional $\bigoplus_{g_1\in G}\, [g_1]\,\R = C_1^{\R} \to \R$.
Somit ist auch hier $f$ eindeutig durch die Werte auf~$G$ bestimmt, aber
zusätzlich muss $\norm{f}_\oper < \infty$ für die Operatornorm von~$f$ gelten.
Bezeichnet $f_0\colon G\to\R$ die zu $f$ assoziierte Funktion auf~$G$, so
überlegt man sich leicht, dass
\[ \norm{f}_\oper = \norm{f_0}_\supr \]
gilt. Die Bedingung, dass $f$ stetig ist, ist also äquivalent dazu, dass
$f_0$ beschränkt ist. Damit erhalten wir einen isometrischen Isomorphismus
zwischen $\Cb^1(G;\R)$ und dem $\R$-Vektorraum aller beschränkten Funktionen
auf~$G$ zusammen mit der Supremumsnorm. Wir unterscheiden daher im Folgenden
(auch wieder etwas schlampig) nicht zwischen den beiden normierten Räumen und
interpretieren $\Cb^1(G;\R)$ dem Kontext entsprechend.

Analog entsprechen beschränkte Funktionen $G^k\to\R$ für $k\in\N$ den Elementen
in $\Cb^k(G;\R)$ und man zeigt auch allgemein leicht, dass für $f\in\Cb^k(G;\R)$
und die entsprechende Funktion $f_0\colon G^k\to\R$ stets
$\norm{f}_\oper = \norm{f_0}_\supr$ gilt.

\chapter{Quasimorphismen und Zusammenhang zu beschränkter Kohomologie}
\begin{thSetup}
    Sei $G$ in diesem Kapitel stets eine Gruppe.
\end{thSetup}

\begin{thDef}[Quasimorphismus, Defekt] \hfill
    \begin{itemize}
        \item
            Ein \emph{Quasimorphismus (auf $G$)} ist eine
            Abbildung $f\colon G\to\R$ mit
            \[ \sup_{g,h\in G} \abs{f(g) + f(h) - f(gh)} \;<\, \infty  . \]

        \item
            Der \emph{Defekt~$D(F)$} eines Quasimorphismus $f\colon G\to\R$
            ist
            \[ D(f) \defeq \inf\bigl\{ c\in\R[\geq0] \Mid
              \sup\nolimits_{g,h\in G}\, \abs{f(g) + f(h) - f(gh)} \leq c \bigr\}
            . \]
    \end{itemize}
\end{thDef}

\begin{BspList}
\item
    Ein Quasimorphismus $G\to\R$ hat genau dann Defekt~$0$, wenn er ein
    Gruppenhomomorphismus $G\to\R$ ist.

\item
    Jede beschränkte Abbildung $f\colon G\to\R$ ist ein
    Quasimorphismus mit $D(f) \leq 3\norm{f}_\supr$.

\item
    Allgemeiner ist für jeden Gruppenhomomorphismus $f\colon G\to\R$ und
    jede beschränkte Abbildung $b\colon G\to\R$ die (punktweise) Summe $f+b$
    ein Quasimorphismus.
\end{BspList}

Das letzte Beispiel gibt Anlass zu folgender Definition:

\begin{thDef}[Trivialer Quasimorphismus]
    Sei $f\colon G\to\R$ ein Quasimorphismus. Dann ist $f$ ein \emph{trivialer
    Quasimorphismus}, falls es einen Gruppenhomomorphismus $f'\colon G\to\R$
    mit $\norm{f-f'}_\supr < \infty$ gibt.
\end{thDef}

Offensichtlich bildet die Menge aller Quasimorphismen (bzw. aller trivialen
Quasimorphismen) einen $\R$-Vektorraum.

\begin{thDef}[Vektorraum aller (trivialen) Quasimorphismen]
    Wir schreiben $\QM(G)$ (bzw. $\QM_0(G)$) für den $\R$-Vektorraum aller
    Quasimorphismen (bzw. aller trivialen Quasimorphismen). 
\end{thDef}

\begin{thProposition}[Zerlegung von \texorpdfstring{$\QM_0$}{QM0}]
    Es gilt:
    \[ \QM_0(G) = \Hom_\Z(G,\R) \oplus \Cb^1(G)  . \]
\end{thProposition}

\begin{proofsketch}
    Die Gleichheit $\QM_0(G) = \Hom_\Z(G,\R) + \Cb^1(G)$ liest man direkt an den
    Definitionen ab und weiter überlegt man sich leicht, dass es keine beschränkten
    Gruppenhomomorphismen $G\to\R$ gibt.
    \\
\end{proofsketch}

\begin{thBeispiel}[Wörterzählende Quasimorphismen]
    Nicht-triviale Beispiele liefern sogenannte \enquote{counting
    quasimorphisms}, welche wir im Folgenden auf Deutsch als
    \emph{wörterzählende Quasimorphismen} bezeichnen. Sei $F$ die freie Gruppe
    über einer Menge~$S$. Elemente aus $F$ seien stets reduzierte Wörter in den
    Erzeugern aus~$S$. Ist nun $w\in F$, so kann man zeigen, dass die Abbildung
    \begin{align*}
        \psi_w\colon F &\to \R  \\
        g &\mapsto (\text{Anzahl der Vorkommen von $w$ in $g$})
                 - (\text{Anzahl der Vorkommen von $w^{-1}$ in $g$})
    \end{align*}
    einen Quasimorphismus auf~$F$ liefert. Einen Beweis dieser Behauptung findet
    man beispielsweise bei
    Löh\cite[Ch.\,2,\;2.5.3,\;Lemma~2.5.11]{lecnotes:loeh:bdcoho}.
    Für eine ausführlichere Behandlung (auch anderer Varianten) wörterzählender
    Quasimorphismen siehe Caligari\cite[Ch.\,2,\;2.3.2]{bookc:calegari09}.
\end{thBeispiel}

\begin{thErinnerDef}[Vergleichsabbildung, $\EHb^\ast$] \hfill
    \begin{itemize}
        \item
            Die Inklusion von Kokettenkomplexen $\Cb^\ast(G;\R) \hookrightarrow
            C^\ast(G;\R)$ induziert eine Abbildung $\Hb^\ast(G;\R) \to
            H^\ast(G;\R)$ in Kohomologie, die sogenannte
            \emph{Vergleichsabbildung}.
        \item
            Es ist
            \[ \EHb^\ast(G;\R)
                \defeq \ker\bigl( \Hb^\ast(G;\R) \to H^\ast(G;\R) \bigr)
            \]
            der (gradweise) Kern der Vergleichsabbildung.
    \end{itemize}
\end{thErinnerDef}

\begin{thSatz}[Zusammenhang zwischen Quasimorphismen und Kohomologie]
    Es gilt:
    \[ \Quot{\QM(G)}{\QM_0(G)} \;\cong\; \EHb^2(G;\R)  . \]
\end{thSatz}

\begin{proof}
    Wir betrachten das folgende Diagramm:
    \[
        \newcommand{\gray}[1]{{\color{black!50}#1}}
        \newcommand{\graylap}[1]{\mathrlap{\gray{#1}}}
        \xymatrix@C=1cm{
            C^1(G;\R) \ar[r]^{\delta^1}
            & C^2(G;\R) \ar[r]^{\pi}
            & H^2(G;\R) \ar@{}[r]|-{\gray=}
                & \graylap{\ker\bigl(\delta^2\bigr)/\im\bigl(\delta^1\bigr)}
            \\
            \QM(G;\R) \ar@{-->}[r]^{\delta} \ar@{ `->}[u]
                \ar@{-->}[rrd]^(.6)\phi
                & \Cb^2(G;\R) \ar[r]^{\pi_\bd} \ar@{ `->}[u]
            & \Hb^2(G;\R) \ar@{}[r]|-{\gray=} \ar[u]_c
                & \graylap{\ker\bigl(\delta^2_\bd\bigr)/\im\bigl(\delta^1_\bd\bigr)}
            \\
            & & \EHb^2(G;\R) \ar@{}[r]|-{\gray=} \ar@{ `->}[u]
                & \graylap{\ker(c)}
            %
            & \hspace*{1.5cm} % alignment hack :/
        }
    \]
    (wobei $c$ die Vergleichsabbildung ist, $\pi$ und $\pi_\bd$ die
    kanonischen Projektionen sind und die linke Inklusion im Sinne
    von \cref{ch3:basics} zu verstehen ist). Sei $f\in\QM(G)$.
    Dann gilt $\delta^1(f) = f\after\partial_2$ und nach Definition von
    $\partial_2$ somit
    \[ (f\after\partial_2)\bigl([g_1|g_2]\bigr) 
        = f(g_2) - f(g_1g_2) + f(g_1)
    \]
    für alle $g_1,g_2\in G$. Nach den Vorüberlegungen in \cref{ch3:basics} gilt
    damit
    \[ \norm[\big]{\delta^1(f)}
        = \sup_{g_1,g_2\in G} \abs{f(g_2) - f(g_1g_2) + f(g_1)}
        \leq D(f) < \infty
    . \]
    Also faktorisiert $\QM(G;\R) \longhookrightarrow C^1(G;\R)
    \overset{\smash{\delta^1}}\longto C^2(G;\R)$ über $\Cb^2(G;\R)$ und
    wir bezeichnen die induzierte Abbildung einfach mit~$\delta$.
    Weil dann auch das äußere Rechteck im obigen Diagramm kommutiert,
    gilt $c\after\pi_\bd\after\delta = 0$. Nach der universellen Eigenschaft
    des Kerns erhalten wir somit eine Abbildung~$\phi$ wie im Diagramm,
    so dass auch das untere Dreieck kommutiert.
    Sei nun $[g]_\bd\in\EHb^2(G;\R)$, d.\,h. mit $[g] = 0 \in H^2(G;\R)$.
    Dann gilt:
    \[ [g] = 0
        \implies g\in\im\bigl(\delta^1\bigr)
        \implies \exists\,\tilde g\in C^1(G;\R)\colon
            \delta^1(\tilde g) = g \in \Cb^2(G;\R)
    . \]
    Solch ein $\tilde g$ ist dann offenbar ein Quasimorphismus. Es folgt die
    Surjektivität von~$\phi$. Sei $f\in\ker(\phi)$, also
    \[ \phi(f) = 0 \in\ker(c) \subset \Hb^2(G;\R)  \qtextq{bzw.}
        \delta(f) \in \im\bigl(\delta^1_\bd\bigr)
    . \]
    Dann gibt es ein $\tilde f\in\Cb^1(G;\R)$ mit $\delta^1_\bd(\tilde f)
    = \delta(f)$ bzw. $\delta^1(f - \tilde f) = 0 \in\Cb^2(G;\R)$.
    Es folgt $f - \tilde f \in\Hom_\Z(G,\R)$ und damit
    \[ f \in \Hom_\Z(G,\R) + \Cb^1(G;\R) = \QM_0(G)  . \]
    Dies zeigt $\ker(\phi) \subset \QM_0(G)$. Sei umgekehrt
    $f = f'+b \in\QM_0(G)$ mit $f'\in\Hom_\Z(G,\R)$ und $b\in\Cb^1(G;\R)$.
    Dann gilt
    \[ \bigl[\delta(f'+b)\bigr]_\bd 
        = \bigl[0 + \delta(b)\bigr]_\bd = 0 \in \Hb^2(G;\R)
    ; \]
    also folgt $\QM_0(G) \subset \ker(\phi)$ und damit Gleichheit. Der
    Homomorphiesatz liefert nun die Behauptung.
    \\
\end{proof}

\begin{thKorollar}
    Verschwindet die zweite Kohomologie $H^2(G;\R)$ von $G$, so gilt
    \[ \Quot{\QM(G)}{\QM_0(G)} \;\cong\; \Hb^2(G;\R)  . \]
\end{thKorollar}

\begin{thBeispiel}[Quasimorphismen auf freien Gruppen]
    Sei $F$ eine freie Gruppe vom Rang mindestens~$2$. Aus dem vorherigen
    Vortrag wissen wir, % TODO: Ist das so?
    dass $\dim_{\R} \Hb^2(G;\R) = \infty$ gilt (oder siehe
    \cref{freep:cohofreegrp}). Außerdem kann man zeigen, dass $H^2(G;\R)$
    verschwindet (siehe zum Beispiel
    Löh\cite[Ch.\,1,\;1.3.4,\;Example~1.3.13]{lecnotes:loeh:bdcoho}).
    Dann folgt aus dem Korollar sofort
    \[ \dim_{\R} \Quot{\QM(G)}{\QM_0(G)} = \infty  , \]
    d.\,h. es gibt unendlich viele (linear unabhängige) nicht-triviale
    Quasimorphismen auf~$F$.
\end{thBeispiel}

\chapter{Ungerade und homogene Quasimorphismen}
\begin{thSetup}
    Sei $G$ auch in diesem Kapitel eine Gruppe.
\end{thSetup}

\begin{thDef}[Ungerader/Homogener Quasimorphismus]
    Sei $f\in\QM(G)$ ein Quasimorphismus.
    \begin{itemize}
        \item
            Wir sagen, $f$ ist \emph{ungerade}, falls
            \[ \forall\,g\in G\colon\quad
                f(g^{-1}) = -f(g)
            \]
            gilt.
            
        \item
            Wir sagen, $f$ ist \emph{homogen}, falls
            \[ \forall\,g\in G,\;k\in\Z\colon\quad
                f(g^k) = k\,f(g)
            \]
            gilt.
    \end{itemize}
\end{thDef}

\begin{thProposition}[Homogenisierung]
    Sei $f\in\QM(G)$. Dann ist die Abbildung
    \[ f\homo \colon G \to \R, \quad
        x\mapsto \lim_{n\to\infty} \frac{f(x^n)}{n}
    \]
    wohldefiniert und ein homogener Quasimorphismus, welchen wir
    \emph{Homogenisierung von $f$} nennen. Außerdem gilt
    \[ \norm{f-f\homo}_\supr \leq D(f)  . \]
\end{thProposition}

Zum Beweis verwenden wir das folgende Lemma:

\begin{thLemma}
    Sei $f\in\QM(G)$, sei $n\in\N$ und seien
    $w_1,\dots,w_n\in G$. Dann gilt für $w\defeq w_1\cdots w_n\in G$:
    \[ \abs[\Big]{ f(w) - \isum^n f(w_i) } \leq (n-1) D(f)  . \]
\end{thLemma}

\begin{thKorollar}
    Es gilt
    \[ \QM(G) = \QM\homo(G) \oplus \Cb^1(G)  , \]
    wobei $\QM\homo(G) \subset \QM(G)$ den Untervekorraum aller homogenen
    Quasimorphismen bezeichnet. Insbesondere gilt auch
    \[ \Quot{\QM\homo(G)}{\Hom_\Z(G,\R)} \;\cong\; E\Hb^2(G;\R)  . \]
\end{thKorollar}

\begin{thKorollar}[Quasimorphismen auf abelschen Gruppen]
    Sei $G$ abelsch. Dann gilt gibt es keine nicht-trivialen
    Quasimorphismen auf~$G$, d.\,h. es gilt $\QM(G) = \QM_0(G)$.
\end{thKorollar}


\appendix
\bibliographystyle{plaindin}
\bibliography{bibsources}

\end{document}





