
\chapter{Vorwort, Notation und Vorbereitungen}
\section{Einleitung}
In den letzten Vorträgen haben wir \emph{beschränkte Kohomologie von Gruppen}
betrachtet und erste Eigenschaften selbiger gesehen. In diesem Skript werden
wir nun sogenannte \emph{Quasimorphismen} einführen und einen Zusammenhang zu
beschränkter Kohomologie beweisen \pcref{qmor:qmcoho}. Dabei verstehen wir unter
einem Quasimorphismus eine Abbildung, die \enquote{bis auf einen beschränkten
Fehler ein Gruppenhomomorphismus ist} \pcref{qmor:def}. Anschließend betrachten wir
gewisse \enquote{Gutartigkeitsbedingungen} von Quasimorphismen \pcref{homo}
und zeigen, wie man Quasimorphismen auf freien Produkten konstruieren kann
\pcref{freep}.

Im Folgenden betrachten wir die rellen Zahlen~$\R$ stets als Gruppe bezüglich
der üblichen Addition auf~$\R$ und sofern nicht anders beschrieben (oder
offensichtlich) schreiben wir die Verknüpfung auf einer Gruppe~$G$ immer
multiplikativ (wobei wir meist keine expliziten Verknüpfungszeichen setzen).

\section{Notation}
In diesem Skript wird folgende Notation verwendet:
\begin{itemize}
    \item
        Sowohl $\subset$ als auch $\subseteq$ stehen für: enthalten oder gleich.
        Echt enthalten wird durch $\subsetneq$ gekennzeichnet.
    
    \item
        Die \emph{natürlichen Zahlen $\N$} beginnen mit $0$.

    \item
        Ist $f\colon X\to\R$ eine Abbildung, so bezeichnet
        $\norm{f}_\supr = \sup_{x\in X}\,\abs{f(x)}$ die
        Supremumsnorm von~$f$.

    \item
        Für einen Ring~$R$ seien $\lCh R$ bzw. $\lCoCh R$
        die Kategorien der Kettenkomplexe bzw. Kokettenkomplexe, jeweils
        $\Z$-indiziert und in $\lMod R$, der Kategorie der Links-$R$-Moduln.
        Außerdem bezeichnet $\lGrad R$ die Kategorie der $\Z$-graduierten
        Links-$R$-Moduln.

    \item
        Sind $V$ und $W$ zwei normierte $\R$-Vektorräume, so ist
        $B(V,W)$ der bezüglich der Opertornorm normierte $\R$-Vektorraum
        aller stetig linearen Operatoren $V\to W$.
\end{itemize}

\section{Grundlegende Definitionen und Vorüberlegungen}
\label{ch3:basics}%
%
Sei $G$ eine Gruppe. Wir bezeichnen den \emph{Bar-Kettenkomplex} von~$G$
mit $C_\ast(G) = (C_\ast,\partial_\ast)$, wobei
\[ C_k(G) = \bigoplus_{g\in G^k} [g_1|\ldots|g_k]\,\Z \]
für alle $k\in\N$ und $C_k(G) = 0$ für $k\in\Z_{<0}$ sowie
\begin{align*}
    \partial_k\colon C_k(G) &\to C_{k-1}(G)
    \\
    [g_1|\ldots|g_k] &\mapsto [g_2|\ldots|g_k]
    + \isum^{k-1} (-1)^i \, [g_1|\ldots|g_ig_{i+1}|\ldots|g_k]
    + (-1)^k \, [g_1|\ldots|g_{k-1}]
\end{align*}
(auf Erzeugern) für alle $k\in\N$. Der \emph{Bar-Kokettenkomplex mit reellen
Koeffizienten} ist der (algebraisch) dualisierte Kokettenkomplex
\[ C^\ast(G;\R) = \Hom_\Z\bigl( C_\ast(G), \R \bigr)  \;\in\lCoCh{\R} , \]
wobei wir die Folge der induzierten Korandoperatoren mit~$\delta$ bezeichnen.
Der \emph{reelle Bar-Kettenkomplex}
\[ C_\ast^{\R}(G)
    \defeq C_\ast(G) \underset{\scriptscriptstyle\Z}\otimes \R
    \;\in\lCh{\R}
\]
wird zusammen mit der $\ell^1$-Norm bezüglich der Basis zu einem
normierten Kettenkomplex. Der topologisch dualisierte Kokettenkomplex
\[ \Cb^\ast(G;\R) \defeq B\bigl( C_\ast^{\R}(G), \R \bigr) \]
ist der \emph{Banach-Bar-Kokettenkomplex} zu~$G$ und
besteht aus Banachräumen bezüglich der Operatornorm und der Folge der
induzierten Korandoperatoren~$\delta_\bd$.

Die \emph{(gewöhnliche) (Gruppen-)Kohomologie} von $G$ (mit reellen
Koeffizienten) ist
\[ H^\ast(G;\R) = H^\ast\bigl( C^\ast(G;\R) \bigr)  , \]
wobei $H^\ast\colon\lCh{\R}\to\lGrad{\R}$ den algebraischen Kohomologiefunktor
bezeichnet. Die \emph{beschränkte (Gruppen-)Kohomologie} von $G$ ist
\[ \Hb^\ast(G;\R) = H^\ast\bigl( \Cb^\ast(G;\R) \bigr)  . \]

Nach Definition gilt $C_1(G) = \bigoplus_{g_1\in G}\, [g_1]\,\Z$. Fassen wir
$G$ vermöge der Inklusion auf die entsprechenden Basiselemente aus $C_1(G)$
als Teilmenge von $C_1(G)$ auf, so ist ein Kokettenelement $f\in C^1(G;\R)$ also 
eindeutig durch die Werte auf $G$ bestimmt. Dies liefert einen
$\R$-Vektorraum-Isomorphismus 
\[ C^1(G;\R) \cong \R^G  \]
zwischen $C^1(G;\R)$ und dem Raum aller $\R$-wertigen Funktionen auf~$G$.
Wir werden daher (etwas schlampig) im Folgenden nicht zwischen den beiden
$\R$-Vektorräumen unterscheiden und $C^1(G;\R)$ jeweils dem Kontext entsprechend
interpretieren.

Ähnlich gehen wir bei $\Cb^1(G;\R)$ vor: Ein Element $f\in\Cb^1(G;\R)$ ist ein
stetig lineares Funktional $\bigoplus_{g_1\in G}\, [g_1]\,\R = C_1^{\R} \to \R$.
Somit ist auch hier $f$ eindeutig durch die Werte auf~$G$ bestimmt, aber
zusätzlich muss $\norm{f}_\oper < \infty$ für die Operatornorm von~$f$ gelten.
Bezeichnet $f_0\colon G\to\R$ die zu $f$ assoziierte Funktion auf~$G$, so
überlegt man sich leicht, dass
\[ \norm{f}_\oper = \norm{f_0}_\supr \]
gilt. Die Bedingung, dass $f$ stetig ist, ist also äquivalent dazu, dass
$f_0$ beschränkt ist. Damit erhalten wir einen isometrischen Isomorphismus
zwischen $\Cb^1(G;\R)$ und dem $\R$-Vektorraum aller beschränkten Funktionen
auf~$G$ zusammen mit der Supremumsnorm. Wir unterscheiden daher im Folgenden
(auch wieder etwas schlampig) nicht zwischen den beiden normierten Räumen und
interpretieren $\Cb^1(G;\R)$ dem Kontext entsprechend.

Analog entsprechen beschränkte Funktionen $G^k\to\R$ für $k\in\N$ den Elementen
in $\Cb^k(G;\R)$ und man zeigt auch allgemein leicht, dass für $f\in\Cb^k(G;\R)$
und die entsprechende Funktion $f_0\colon G^k\to\R$ stets
$\norm{f}_\oper = \norm{f_0}_\supr$ gilt.
