\section{Ungerade und homogene Quasimorphismen}
\begin{thDef}[Ungerader/Homogener Quasimorphismus]
    Sei $f\in\QM(G)$.
    \begin{itemize}
        \item
            Wir sagen, $f$ ist \emph{ungerade}, falls gilt:
            $\forall\,g\in G\colon\; f(g^{-1}) = -f(g)$.
            
        \item
            Wir sagen, $f$ ist \emph{homogen}, falls gilt:
            $\forall\,g\in G,\,k\in\Z\colon\;
                f(g^k) = k \mkern2mu f(g)$.
    \end{itemize}
\end{thDef}

\begin{thAufgabe}
    Zeige: Ein homogener Quasimorphismus~$f$ auf $G$ ist konstant auf
    Konjugationsklassen, d.\,h. für alle $g,h\in G$ gilt $f(h) = f(ghg^{-1})$.
\end{thAufgabe}

\begin{thProposition}[Homogenisierung]
    \label{homo:homo}%
    %
    Sei $f\in\QM(G)$. Dann ist die \emph{Homogenisierung $f\homo$ von $f$},
    gegeben durch
    \[ f\homo \colon G \to \R, \quad
        x\mapsto \lim\nolimits_{n\to\infty}\, f(x^n)/n
    , \]
    wohldefiniert und ein homogener Quasimorphismus mit
    $\norm{f-f\homo}_\supr \leq D(f)$.
\end{thProposition}

\begin{thLemma} \label{homo:qmprod}
    Sei $f\in\QM(G)$, sei $n\in\N[\geq1]$ und seien
    $w_1,\dots,w_n\in G$. Dann gilt für $w\defeq w_1\cdots w_n\in G$:
    \quad
    $\abs[\big]{ f(w) - \sum\nolimits_{i=1}^n f(w_i) } \leq (n-1)\cdot D(f)$.
\end{thLemma}

\begin{thAufgabe}
    Beweise und benutze \cref{homo:qmprod}, um \cref{homo:homo} zu zeigen.
\end{thAufgabe}
