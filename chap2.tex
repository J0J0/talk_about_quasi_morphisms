\chapter{Ungerade und homogene Quasimorphismen} \label{homo}
\begin{thSetup}
    Sei $G$ auch in diesem Kapitel eine Gruppe.
\end{thSetup}

\begin{thDef}[Ungerader/Homogener Quasimorphismus]
    Sei $f\in\QM(G)$.
    \begin{itemize}
        \item
            Wir sagen, $f$ ist \emph{ungerade}, falls gilt:
            \[ \forall\,g\in G\colon\quad
                f(g^{-1}) = -f(g)
            . \]
            
        \item
            Wir sagen, $f$ ist \emph{homogen}, falls gilt:
            \[ \forall\,g\in G,\;k\in\Z\colon\quad
                f(g^k) = k \mkern2mu f(g)
            . \]
            
        \item
            Den Untervektorraum von $\QM(G)$, der alle homogenen Quasimorphismen
            umfasst, bezeichnen wir mit $\QM\homo(G)$.
    \end{itemize}
\end{thDef}

\begin{thProposition}[Homogenisierung]
    \label{homo:homo}%
    %
    Sei $f\in\QM(G)$. Dann ist die Abbildung
    \[ f\homo \colon G \to \R, \quad
        x\mapsto \lim_{n\to\infty} \frac{f(x^n)}{n}
    \]
    wohldefiniert und ein homogener Quasimorphismus, welchen wir
    \emph{Homogenisierung von~$f$} nennen. Außerdem gilt
    \[ \norm[\big]{f-f\homo}_\supr \leq D(f)  . \]
\end{thProposition}

\begin{thBemerkung} \label{homo:homoisproj}
    Ist $f\in\QM(G)$ bereits homogen, so sehen wir sofort anhand der Definition
    von $f\homo$ die Gleichheit $f\homo = f$ ein, das heißt Homogenisieren ist die
    Identität auf $\QM\homo(G) \subset \QM(G)$. Außerdem folgt ebenfalls sofort
    aus der Definition, dass Homogenisieren $\R$-linear ist, womit
    \[ {\scdot}\homo\colon \QM(G) \to \QM(G) \]
    eine lineare Projektion auf $\QM\homo(G) \subset \QM(G)$ ist.
\end{thBemerkung}

Zum Beweis von \cref{homo:homo} verwenden wir das folgende Lemma:

\begin{thLemma} \label{homo:qmprod}
    Sei $f\in\QM(G)$, sei $n\in\N[\geq1]$ und seien
    $w_1,\dots,w_n\in G$. Dann gilt für $w\defeq w_1\cdots w_n\in G$:
    \[ \abs[\Big]{ f(w) - \isum^n f(w_i) } \leq (n-1)\cdot D(f)  . \]
\end{thLemma}

\begin{proof}
   Wir führen Induktion über $n$. Der Fall $n=1$ ist klar. Sei also
   $n\in\N[\geq2]$. Dann gilt:
   \begin{align*}
       \abs[\Big]{ f(w) - \isum^n f(w_i) }
       &= \abs[\Big]{
           f(w) - f(w_1\cdots w_{n-1}) + f(w_1\cdots w_{n-1})
           - \isum^{n-1} f(w_i) - f(w_n)
          }
       \\
       &\leq
       \abs[\big]{f(w) - f(w_1\cdots w_{n-1}) - f(w_n)}
       + \abs[\Big]{f(w_1\cdots w_{n-1}) - \isum^{n-1} f(w_i)}
       \\
       &\overset{\text{\clap{\tiny IV}}}\leq
           D(f) + (n-1-1) \cdot D(f)
        = (n-1) \cdot D(f)
        \\
        &\qedhere
   \end{align*}
\end{proof}

\begin{proof}[Beweis von \cref{homo:homo}]
    Wir zeigen zuerst, das der Grenzwert für alle $x\in G$ existiert. Sei also
    $x\in G$ und seien $n,m\in\N[\geq1]$. Nach \cref{homo:qmprod} gilt
    \[  \abs[\big]{f(x^{nm}) - n f(x^m)}
         \leq (n-1) D(f)
       \qtextq{sowie}
       \abs[\big]{f(x^{nm}) - m f(x^n)}
         \leq (m-1) D(f)
    . \]
    Mithilfe der Dreiecksungleichung erhalten wir daraus
    \[ \abs[\big]{nf(x^m) - mf(x^n)}
        \leq (n+m-2) D(f)
    . \]
    Daraus folgt
    \[ \abs*{ \frac{f(x^m)}{m} - \frac{f(x^n)}{n} }
        \leq \left( \frac{1}{m} + \frac{1}{n} - \frac{2}{mn} \right) D(f)
        \;\xrightarrow[m,n\to\infty]{}\; 0
    , \]
    d.\,h. $\bigl( f(x^n)/n \bigr)_{n\in\N[\geq1]}$ ist eine Cauchyfolge in~$\R$
    und damit auch konvergent. Aus der Definition von $f\homo$ folgt sofort
    $f\homo(e) = 0$ für das neutrale Element $e\in G$. Also gilt
    \begin{align*}
        \abs[\big]{ f\homo(x^{-1}) + f\homo(x) }
        &= \abs[\big]{ f\homo(x^{-1}) + f\homo(x) - f\homo(e) }
        = \lim_{n\to\infty} \, \frac{1}{n} \,
            \abs[\big]{ f(x^{-n}) + f(x^n) - f(x^{-n}x^n) }
        \\
        &\leq \lim_{n\to\infty} \, \frac{1}{n} \, D(f) = 0
    , \end{align*}
    d.\,h. $f\homo$ ist schon einmal ungerade. Für alle $x\in G$ und alle
    $m\in\N[\geq1]$ gilt außerdem
    \[ \abs[\big]{f\homo(x^m) - mf\homo(x)}
        = \lim_{n\to\infty} \,
        \frac{1}{n} \, \abs[\big]{ f(x^{mn}) - mf(x^n) }
        \leq \lim_{n\to\infty} \frac{m-1}{n} \, D(f) = 0
    , \]
    d.\,h. zusammen mit dem vorherigen Argument folgt: 
    $f\homo$ ist wirklich homogen. Für alle $x\in G$ gilt weiterhin
    \[ \abs[\big]{f(x)-f\homo(x)}
        = \lim_{n\to\infty} \,
        \frac{1}{n} \, \abs[\big]{nf(x)-f(x^n)}
        \leq \lim_{n\to\infty} \frac{n-1}{n} \, D(f) = D(f)
    , \]
    was die letzte Behauptung zeigt, und damit insbesondere, dass $f\homo$ ein
    Quasimorphismus ist.
    \\
\end{proof}

\begin{thKorollar} \label{homo:decomp}
    Es gilt
    \[ \QM(G) = \QM\homo(G) \oplus \Cb^1(G;\R)  . \]
    Insbesondere gilt auch
    \[ \Quot{\QM\homo(G)}{\Hom_\Z(G,\R)} \;\cong\; \EHb^2(G;\R)  . \]
\end{thKorollar}

\begin{proof}
    Sei $H\defeq {\cdot\,}\homo$. Aus \cref{homo:homoisproj} und einem einfachen
    Resultat aus der linearen Algebra folgt:
    \[ \QM(G) = \im(H) \oplus \im\bigl(\id_{\QM(G)}-H\bigr) . \]
    Weil die Homogenisierung eines beschränkten Quasimorphismus offensichtlich
    trivial ist, gilt $\im(\id_{\QM(G)}-H) = \Cb^1(G;\R)$ und es folgt die
    erste Behauptung.
    Die zweite Behauptung ist eine direkte Folgerung aus dem ersten Teil,
    \cref{qmor:decompQM0} und \cref{qmor:qmcoho}.
    \\
\end{proof}

\begin{thKorollar}[Quasimorphismen auf abelschen Gruppen]
    Sei $G$ abelsch. Dann gibt es keine nicht-trivialen
    Quasimorphismen auf~$G$, d.\,h. es gilt $\QM(G) = \QM_0(G)$.
\end{thKorollar}

\begin{proof}
    Sei $f\in\QM\homo(G)$ und seien $x,y\in G$. Dann gilt
    $(xy)^n = x^ny^n$ für alle $n\in\N$, also:
    \[  \abs[\big]{ f(x) + f(y) - f(xy) }
        = \frac{1}{n} \, \abs[\big]{ f(x^n) + f(y^n) - f(x^ny^n) }
        \leq \frac{1}{n} \, D(f)
        \;\xrightarrow[n\to\infty]{}\; 0
    . \]
    Also ist $f$ schon ein Gruppenhomomorphismus und es gilt somit
    $\QM\homo(G) = \Hom_\Z(G,\R)$. Mit \cref{homo:decomp} folgt
    die Behauptung.
    \\
\end{proof}
