\chapter{Ungerade und homogene Quasimorphismen}
\begin{thSetup}
    Sei $G$ auch in diesem Kapitel eine Gruppe.
\end{thSetup}

\begin{thDef}[Ungerader/Homogener Quasimorphismus]
    Sei $f\in\QM(G)$.
    \begin{itemize}
        \item
            Wir sagen, $f$ ist \emph{ungerade}, falls gilt:
            \[ \forall\,g\in G\colon\quad
                f(g^{-1}) = -f(g)
            . \]
            
        \item
            Wir sagen, $f$ ist \emph{homogen}, falls gilt:
            \[ \forall\,g\in G,\;k\in\Z\colon\quad
                f(g^k) = k \mkern2mu f(g)
            . \]
    \end{itemize}
\end{thDef}

\begin{thProposition}[Homogenisierung]
    Sei $f\in\QM(G)$. Dann ist die Abbildung
    \[ f\homo \colon G \to \R, \quad
        x\mapsto \lim_{n\to\infty} \frac{f(x^n)}{n}
    \]
    wohldefiniert und ein homogener Quasimorphismus, welchen wir
    \emph{Homogenisierung von~$f$} nennen. Außerdem gilt
    \[ \norm[\big]{f-f\homo}_\supr \leq D(f)  . \]
\end{thProposition}

Zum Beweis verwenden wir das folgende Lemma:

\begin{thLemma}
    Sei $f\in\QM(G)$, sei $n\in\N$ und seien
    $w_1,\dots,w_n\in G$. Dann gilt für $w\defeq w_1\cdots w_n\in G$:
    \[ \abs[\Big]{ f(w) - \isum^n f(w_i) } \leq (n-1) D(f)  . \]
\end{thLemma}

\begin{thKorollar}
    Es gilt
    \[ \QM(G) = \QM\homo(G) \oplus \Cb^1(G)  , \]
    wobei $\QM\homo(G) \subset \QM(G)$ den Untervekorraum aller homogenen
    Quasimorphismen bezeichnet. Insbesondere gilt auch
    \[ \Quot{\QM\homo(G)}{\Hom_\Z(G,\R)} \;\cong\; \EHb^2(G;\R)  . \]
\end{thKorollar}

\begin{thKorollar}[Quasimorphismen auf abelschen Gruppen]
    Sei $G$ abelsch. Dann gilt gibt es keine nicht-trivialen
    Quasimorphismen auf~$G$, d.\,h. es gilt $\QM(G) = \QM_0(G)$.
\end{thKorollar}
